\documentclass[12pt, letterpaper]{article}
\usepackage[utf8]{inputenc}
\usepackage[portuguese]{babel}
\usepackage{graphicx}
\usepackage{amsmath}
\usepackage{geometry}
\usepackage{caption}
\usepackage{pgfplots}
\usepackage{tikz}
\usepackage{ulem}
\usepackage{indentfirst}
\usepackage{circuitikz}
\usepackage{float}
\usepackage{gensymb}
\pgfplotsset{width=10cm, compat=1.18}
\geometry{
	paper=a4paper, 
	inner=3cm,
	outer=3cm,
	bindingoffset=.5cm, 
	top=2cm, 
	bottom=2cm, 
}
\graphicspath{ {./imagens/} }
\newcommand{\source}[1]{\caption*{Fonte: {#1}} }
\usepackage{xcolor}
\colorlet{veccol}{green!45!black}
\colorlet{myred}{red!90!black}
\colorlet{myblue}{blue!90!black}
\colorlet{mypurple}{blue!50!red!80!black!80}
\tikzstyle{vector}=[->,very thick,veccol]
\usetikzlibrary{arrows.meta}
\tikzstyle{thin arrow}=[dashed,thin,-{Latex[length=4,width=3]}]
\usetikzlibrary{calc}
\tikzset{>=latex} % for LaTeX arrow head

\begin{document}

\begin{titlepage}
    \begin{center}
        \Large
        \vspace*{1cm}

        \includegraphics[scale=0.75]{imagens/ifsc_logo.png} \\
        \Huge
        \textbf{Universidade de São Paulo} \\
        \huge
        Estática
        
        \Large
        Laboratório de física I
        \vspace{4cm}
        \Large
        
    \end{center}
    
    \begin{flushright}
        \Large
        \textbf{Alunos}\\
        \large
            Luis Eduardo Aires Coimbra - n° 15472565\\
            Marco Antônio Bicalho de Oliveria - n° 15474741\\
            Mateus Alves Martins Claudino - n° 15478791
    \end{flushright}
    
    \vspace*{\fill}
    \centering \large São Carlos \\ 2024
    
\end{titlepage}
 \tableofcontents
 \newpage

\section{Introdução}
A Estática é a área da Física que estuda o equilíbrio de corpos rígidos. Equilíbrio significa ausência de aceleração e, portanto, velocidades de translação e rotação constantes. Nos problemas de equilíbrio estático é considerado também que as velocidades são nulas. As condições de equilíbrio de um sistema estão determinadas pelas Leis de Newton da Mecânica (1) e (2) [1]: 
\begin{equation}
\sum\limits_{i}^{\mbox{}}\vec{F_i} = 0
\end{equation}
\begin{equation}
\sum\limits_{i}^{\mbox{}}\vec{F_i}\times\vec{r_i} = 0
\end{equation}


Nessas equações, $\vec{F_i}$ são as forças externas atuando sobre o sistema e
$\vec{r_i}$, os pontos de aplicação de cada uma, quando o sistema é extenso, como,
por exemplo, um corpo rígido.
A condição (1) garante que o centro de massa
do sistema tem aceleração nula; se inicialmente está em repouso, então se
manterá nesse estado e não haverá translação. A condição (2) é necessária
para garantir que o sistema não vai se acelerar angularmente e, portanto, não
vai rotar. A condição (2) é imprescindível para determinar o equilíbrio de corpos
rígidos, por exemplo, em projetos de construções civis, motores ou aeronaves.
Em problemas de estática mais simples, para garantir o equilíbrio translacional
do centro de massa do corpo, muitas vezes é suficiente que somente a
condição (1) seja satisfeita [1].
\subsection{Diagrama de forças}

Ao analisar a condição de equilíbrio de um sistema físico, é comum representar as forças externas atuando sobre o sistema concentradas em um único ponto, que pode ser a posição de uma massa pontual ou o centro de massa de um corpo extenso. Essa representação gráfica das forças é conhecida como diagrama de forças de corpo isolado. Nesse diagrama, a força resultante da soma vetorial das forças é nula, refletindo o estado de equilíbrio do sistema.

Um exemplo ilustrativo é mostrado na figura (1), onde um corpo extenso está em equilíbrio sobre um plano inclinado. Nessa situação, três forças estão atuando sobre o corpo: o peso, representado no centro de massa do corpo; a força de reação do plano, aplicada perpendicularmente sobre a face do corpo; e a força de atrito, aplicada paralelamente à face do corpo. Na figura, são apresentados os dois tipos de diagramas de forças mencionados.

No diagrama de corpo isolado, as forças são representadas como vetores que emanam do centro de massa do corpo rígido. Essa representação simplifica a análise ao concentrar todas as forças em um único ponto. A condição de equilíbrio é então expressa pela nulidade da força resultante, o que implica que a soma vetorial de todas as forças é zero.


\begin{center}
    

\begin{tikzpicture}[>=stealth]
    % Definindo os vetores
    \draw[->, thick, red] (0,0) -- (0,-4.1) node[midway, below] {$\mathbf{F_1}$};
    \draw[->, thick, blue] (0,-4.1) -- (2.66,-2.44) node[midway, right] {$\mathbf{F_2}$};
    \draw[->, thick, green] (2.66,-2.44) -- (0,0) node[midway, above] {$\mathbf{F_3}$};
    
    % Adicionando ângulos
    \draw (0,0) arc (0:26.57:0.5) node[midway, right] {$\theta_1$};
    \draw (0,-4.1) arc (26.57:63.43:0.5) node[midway, above] {$\theta_2$};
    \draw (2.66,-2.44) arc (153.43:180:0.5) node[midway, left] {$\theta_3$};
\end{tikzpicture}

\begin{tikzpicture}[>=stealth]
    % Definindo os vetores
    \draw[->, thick, red] (0,0) -- (0,-4.1) node[midway, below] {$\mathbf{F_1}$};
    \draw[->, thick, blue] (0,0) -- (2.66,1.66) node[midway, right] {$\mathbf{F_2}$};
    \draw[->, thick, green] (0,0) -- (-2.7,2.43) node[midway, above] {$\mathbf{F_3}$};
    
    % Adicionando ângulos
    \draw (0,0) arc (0:26.57:0.5) node[midway, right] {$\theta_1$};
    \draw (0,0) arc (26.57:63.43:0.5) node[midway, above] {$\theta_2$};
    \draw (0,0) arc (153.43:180:0.5) node[midway, left] {$\theta_3$};
\end{tikzpicture}

\begin{tikzpicture}
    % Desenho do plano inclinado
    \draw (0,0) -- (4,0);
    \draw (0,0) -- (60:4);
    \draw[dashed] (0,0) -- (0,2);
    \draw[dashed] (4,0) -- (4,2);
    \draw[dashed] (0,2) -- (4,2);
    \draw[dashed] (60:4) -- (60:2);
    \draw[dashed] (0,0) -- (0,-0.5);
    
    % Desenho do bloco
    \draw[fill=gray] (2,1) rectangle (3,1.5) node[midway] {$m$};
    
    % Ângulo
    \draw (0.75,0) arc (0:30:0.75) node[midway, right] {$\theta$};
    
    % Diagrama de forças
    \begin{scope}[shift={(5,1)}]
        % Vetores de força
        \draw[->, thick, blue] (0,0.75) -- (0,2) node[right] {$\mathbf{F_{\text{grav}}}$};
        \draw[->, thick, red] (-0.5,0.25) -- (-1.5,0.25) node[left] {$\mathbf{F_{\text{atr}}}$};
        \draw[->, thick, green] (0.5,0.25) -- (1.5,0.25) node[right] {$\mathbf{N}$};
        \draw[->, thick, purple] (0,-0.25) -- (0,-1.5) node[right] {$\mathbf{P}$};
    \end{scope}
\end{tikzpicture}
\end{center}
Diante dessa análise vetorial é possivél fazer o uso da estrutura matemática da Lei dos Senos a fim de simplificar a análise vetorial de decomposição, conforme a seguinte equação(3):


\begin{equation}
    \frac{F_{at}}{\sin{\theta_{1}}} = \frac{P}{\sin{\theta_{2}}} = \frac{N}{\sin{\theta_{3}}}
\end{equation}
    


\subsection{Forças de Atrito}
Quando dois corpos são colocados em contato, forças em escala
molecular são mutuamente exercidas entre as superfícies. As forças de reação
de contato que aparecem quando um objeto exerce pressão sobre outro são o
exemplo mais direto desse fenômeno. Essas forças têm origem na repulsão
entre os elétrons nas duas superfícies. Existem também forças atrativas na
escala molecular, como as forças de van der Waals, que tendem a dificultar o
deslizamento das superfícies e constituem a origem das forças de atrito [1].


Matematicamente, essas forças podem ser representadas por um vetor $\vec{F_a}$ sempre contrário à tendência de movimento do objeto, de modo que, no repouso, elas aumentam de maneira proporcional à força $\vec{F}$ aplicada ao corpo até um limite determinado por $\vec{F_a}_{\text{max}}$.Ao equacionar obtém-se (4) :
\begin{equation}
F \leq {F_a}_{\text{max}} \hspace{0.2cm} com  \hspace{0.2cm} {F_a}_{\text{max}} = \mu_eN
\end{equation}
em que $\mu_e$ é o coeficiente de atrito estático, que depende das duas superfícies
em contato, e N a normal entre o corpo e superfície.A resposta do atrito, diante da intensidade da força de força $\vec{F}$,
está representada graficamente na figura (2)[1]:
\includegraphics[width = 1\linewidth]{imagens/image.png}

(COLOCAR F NO LUGAR DE T!!)

Acima do limite máximo, quando $F > {F_a}_{\text{max}}$, as forças atrativas
intermoleculares são vencidas e as superfícies começam a deslizar. Nessas
condições, o atrito decai bruscamente e assume um valor aproximadamente
constante (5) [1] :
\begin{equation}
 F_a = \mu_cN   
\end{equation}
esse é o regime de atrito cinético, sendo que $\mu_c$ é o
coeficiente de atrito cinético ou dinâmico entre as superfícies. 

\section{Objetivos}
Analisar e estudar os casos de equilíbrio estático em três experimentos distintos, calcular as forças de tração nas cordas nos dois primeiros experimentos e o coeficiente de atrito estático entre o objeto e o plano inclinado no segundo experimento. Com isso, aprender sobre diferentes casos de equilíbrio, forças de diferentes naturezas e entender a aplicação do diagrama de forças.


\section{Materiais}
\begin{enumerate}
   \item Aste metálica
    \item Balança analítica. Modelo: AUW220D (Shimadzu)
    \item Balde 
    \item Ganchos metálicos
    \item Linha de costura simples
    \item Massas Cilíndricas
    \item Massas em formas de disco
    \item Paralelepípedo reto retângulo com uma face lisa e a oposta rugosa
    \item Plano Inclinado variável
    \item Polias fixas
    \item Régua com suporte variável
    \item Régua de plástico comum
    \item Suporte magnético para transferidor.
    \item Tábua de apoio para as polias magnético
    \item Tesoura
    \item Transferidor, Modelo: Deselec n° 8112
\end{enumerate}

\section{Metodologia}
\subsection{Experimento 1}
    Neste experimento, o grupo utilizou um sistema composto por duas polias e três cordas amarradas em um ponto para determinar a força de tração no caso de equilíbrio estático entre três massas penduradas em cada corda.

    Primeiramente, nas pontas de duas das cordas, foram apoiadas as massas $ m_{1}$ e $m_{3}$, e as cordas foram passadas sobre as polias para poderem se movimentar. Após a fixação dessas duas massas, uma terceira massa, $m_{2}$, foi pendurada na corda central para estabelecer a situação de equilíbrio desejada no ponto de amarração, conforme mostrado na figura (3).
    \begin{figure}[!h]
        \centering
        \includegraphics[width=0.4\linewidth]{imagens/sistema_de_polias.png}
        \caption{Enter Caption}
        \label{fig:enter-label}
    \end{figure}
    
    Em seguida, aguardou-se algum tempo para que o sistema ficasse em equilíbrio, e um transferidor foi utilizado tendo o ponto de amarração como origem, e foi mensurado assim os ângulos $\alpha$ e $\gamma$, como esquematizado na figura (3). Com isso, foi desenhado o diagrama de forças e o diagrama de corpo livre para o sistema em um papel milimetrado.
    
    Utilizando como base a lei dos senos, obteve-se a relação entre as forças de tração em cada corda na situação de equilíbrio, figura (5), dada pelas equações (6) e (7).
    \begin{equation}
        T_{AB} = \frac{\sin(\alpha)m_{2}g}{\sin({180 \degree - \alpha - \gamma})}
    \end{equation}
    \begin{equation}
        T_{AC} = \frac{\sin(\gamma)m_{2}g}{\sin(180 \degree - \alpha - \gamma)}
    \end{equation}

    Para as massas $m_{1}$ e $m_{3}$, foi utilizado o diagrama de corpo livre para cada uma, e foram obtidas as equações (8) e (9) na situação de equilíbrio:
    \begin{equation}
        m_{1} = \frac{T_{AC}}{g} 
    \end{equation}
    \begin{equation}
        m_{2} = \frac{T_{AB}}{g}
    \end{equation}

    Com isso, foi discutida a relação entre os valores encontrados e uma possível situação de forças de atrito entre as polias e as cordas. 
    
\subsection{Experimento 2}
    No segundo experimento, foi utilizada uma corda presa a uma haste e a um apoio móvel, que estava sobre uma régua na vertical, juntamente com uma massa presa a haste em questão para que fosse analisada a situação de ruptura do fio, como esquematizado na figura (4), em que C é o apoio móvel, B a base da régua e A a extremidade da haste:
        \begin{figure}[!h]
            \centering
            \includegraphics[width=0.2\linewidth]{imagens/experimento2.png}
            \caption{Esquema do experimento dois}
            \label{fig:enter-label}
        \end{figure}
    
    Inicialmente, foram presas cordas entre a extremidade da haste e o gancho de massas e também entre a extremidade da haste e um apoio móvel próximo a uma régua. Após isso, subiu-se o apoio móvel até o topo da régua para segurar a haste.

    Após isso, foi mensurada com a ajuda de outra régua o valor do comprimento $L$ do barbante preso ao apoio móvel. Em seguida, foi adicionada uma massa $m_{4}$ ao gancho para que a corda fosse tensionada, e foi medido novamente o valor $L$ do fio, pois com o aumento da tensão esse valor fica ligeiramente maior.

    Posteriormente, o apoio foi descido cuidadosamente até o instante de ruptura e obtido o valor apontado pela régua para a altura do apoio móvel $\overline{\rm CB}$, processo esse repetido cinco vezes e montado na tabela (1).

    Com isso, foi obtido o valor da tração no fio dado pelo resultado (10):

    
    
    \begin{equation}
        T = \frac{mgL}{\overline{\rm CB}}
    \end{equation}
    Também foi calculado a incerteza da tração, através do desvio médio, sendo obtida a equação (11):
    \begin{equation}
        \Delta T = \frac{\sum |T_{i} - \bar T|}{N}
    \end{equation}

    Num segundo momento, foi utilizada a mesma haste e o mesmo barbante porém agora preso na vertical ao gancho e à haste, e foi adicionado aos poucos pequenas massas de valores diferentes, até que o peso das massas fosse suficiente para romper o fio.

    No momento de ruptura, as massas utilizadas foram medidas numa balança, esse procedimento foi repetido cinco vezes e reportado na tabela (2). Cada uma das outras medidas foi feita adicionando-se massas cada vez menores, e com isso, o ponto de ruptura foi mais aproximado e foi calculado o valor médio do peso necessário para romper o fio.

    Após isso, o valor da tração de rompimento do fio foi obtido pela equação (12) com base na equação (1):
    \begin{equation}
        T = mg
    \end{equation}
    Em seguida, ao final, foi calculada a incerteza da tração na situação em questão, da mesma forma expressa na equação (11).
\subsection{Experimento 3}
Primeiramente, a partir do diagrama de corpo livre de um bloco em repouso em um plano inclinado com atrito e a equação (4), obteve-se o resultado (13):
\begin{equation}
\mu_e = \tan{\theta}
\end{equation}
em que $\mu_e$ é o coeficiente de atrito estático, e $\theta$ o ângulo entre  a horizontal e o plano inclinado estudado.


Posteriormente, com um bloco de madeira com duas faces de superfícies distintas,uma lisa e uma rugosa,escolheu-se uma dessas faces e colocou-se esse bloco em um plano inclinado com ângulo $\theta$ varíavel, partindo de $\theta$ = $0$ até o ângulo no qual o bloco passou a deslizar sobre o plano inclinado. O resultado desse ângulo, em graus, foi então anotado com a ajuda de um transferidor,repetindo-se esse experimento um total de vinte vezes.


Após isso, calculou-se o $\mu_e$ para cada um desses ângulos e calculou-se a média $\bar\mu$ desses resultados com sua devida incerteza, como essa sendo igual ao desvio médio calculado por (14):
\begin{equation}
        \Delta\mu_e = \frac{\sum |\mu_{i} - \bar \mu_e|}{N}
    \end{equation}
em que N é igual ao número de medidas realizadas

Com isso, escolheu-se a outra face do bloco e o experimento pode ser realizado novamente, obtendo-se um novo $\mu_e$ com outra incerteza.

Finalmente,comparou-se os coeficientes de atrito obtidos com a tabela (), para chegar a uma conclusão sobre as superfícies em contato.
\section{Resultados}
\subsection{Experimento 1}
    Após pesagem das massas utilizadas no experimento 1, foram obtidos os resultados (15), (16), (17):
    \begin{equation}
        m_{1} = 72,70 \pm 0,01g
    \end{equation}
    \begin{equation}
        m_{2} = 82,60 \pm 0,01g
    \end{equation}
    \begin{equation}
        m_{3} = 62,90 \pm 0,01g
    \end{equation}

    Com as massas posicionadas foram esperados poucos segundos até a situação de equilíbrio se estabelecer, e com um transferidor foram obtidos os valores para os ângulos $\alpha$ e $\gamma$ dados pelos resultados (18) e (19):
    \begin{equation}
        \alpha = 48 \degree \pm 1 \degree
    \end{equation}
    \begin{equation}
        \gamma = 58 \degree \pm 1 \degree
    \end{equation}

    Com os ângulos e as massas, foi construído o diagrama de forças no ponto de encontro das três cordas e também o triângulo de forças para tal ponto, resultando na figura ():

    "diagrama e triangulo no papel milimetrado"
    \begin{figure}[!h]
        \centering
        \begin{tikzpicture}[scale=1.5]
            % Definição dos vetores
            \draw[myred, ->] (0,0) -- (1.4,0.7) node[right] {$\mathbf{F_1}$};
            \draw[myblue, ->] (0,0) -- (270:1) node[left] {$\mathbf{F_2}$};
            \draw[mypurple, ->] (0,0) -- (150:1.5) node[left] {$\mathbf{F_3}$};
            \draw[dashed] (0,0) -- (0, 1);
            
            % Ângulos
            \draw (-0.38,0.22) arc (150:90:0.4) node[midway,left] {$\alpha$};
            \draw (0.3,0.12) arc (1:90:0.3) node[midway,right] {$\gamma$};
        \end{tikzpicture}
        \caption{Diagrama de forças do ponto de amarração}
        \label{fig:diagrama_forcas}
    \end{figure}

    Foi utilizada as equações (6) e (7) para calcular as trações e o valor dado pelo resultado (16). Assim, foram obtidas as relações (20) e (21):
    \begin{equation}
        T_{AB} = 626,4420121 mN
    \end{equation}
    \begin{equation}
        T_{AC} = 714,8713649 mN
    \end{equation}
    Em seguida, foi calculada a incerteza da tração AB obtendo-se a equação (22) e a da tração AC obtendo-se a equação (23):
    \begin{equation}
        \Delta T_{AB} = 16,19064047 mN
    \end{equation}
    \begin{equation}
        \Delta T_{AC} = 15,03832292 mN
    \end{equation}
    Dessa forma, foram obtidos os resultados (24) e (25) para as trações:
    \begin{equation}
        T_{AB} = 630 \pm 20 mN 
    \end{equation}
    \begin{equation}
        T_{AC} = 710 \pm 20 mN
    \end{equation}

    Usando como base as equações (8) e (9), foram calculados os resultados (26) e (27) para os valores indiretos das massas $m_{1}$ e $m_{3}$:
    \begin{equation}
        m_{1} = 72,87169877 g
    \end{equation}
    \begin{equation}
        m_{3} = 63,85 g
    \end{equation}

    Após isso, foram calculadas suas respectivas incertezas, obtendo-se os resultados (28) e (29):
    \begin{equation}
        \Delta m_{1} = 1,532958504 g
    \end{equation}
    \begin{equation}
        \Delta m_{3} = 1,650422066 g
    \end{equation}

    Logo, sendo reportadas as equações (30) e (31) para as massas na medida indireta:
    \begin{equation}
        m_{1} = 72,9 \pm 2 g
    \end{equation}
    \begin{equation}
        m_{3} = 63,9 \pm 2 g
    \end{equation}

    Finalmente, foi feita a equivalência para os valores obtidos direta e indiretamente para as massas, gerando os resultados (32) e (33):
    \begin{equation}
        |72,87169877 - 72,7| < 2|0,01 + 1,532958504| \Rightarrow 0,17169877 < 3,085917008
    \end{equation}
    \begin{equation}
        |63,85 - 62,9| < 2|0,01 + 1,650422066| \Rightarrow 0,95 < 3,320844133
    \end{equation}

    Resultando então em massas equivalentes.
\subsection{Experimento 2}

Nesse experimento, o grupo mediu os comprimentos $L_{i}$ e $\overline{\rm CB}_{i}$ para tensionar o fio até a ruptura por meio da carga $m$ no arranjo experimental da figura ().

Os dados coletados estão descritos na equação (34) e na tabela (1):

\begin{equation}
    m= (543 \pm 0,01)g
\end{equation}

\begin{table}[!h]
 \centering
        \caption{Relação das medidas $L_{i}$ e $BC_{i}$}
         \begin{tabular}{| c | c | c |}
           \hline
             $i$  & $\overline{\rm CB}_{i}[cm](\pm0,1cm)$ & $L_{i}[cm](\pm0,1cm)$  \\
            \hline
               1   & 11,1  &  18,0  \\
           \hline    
            2   &   12,3 &  18,5 \\
           \hline    
            3   &   11,5  &  18,8 \\
           \hline    
               4   & 12,5 &   18,8 \\
           \hline    
               5  &  13,0  & 18,6\\
        
             
             \hline 
        \end{tabular} 
        \label{tab:2}

\end{table}

Depois, a equação (10) foi utilizada para calcular a tração da primeira medição descrita na equação (35):

\begin{equation}
    T= 8,63810270 N
\end{equation}

Dessa maneira, prosseguiu-se o procedimento para calcular uma tração média das cinco medições e seu erro descrito usando a equação (11). Tais dados geraram as equações (36) e (37): 
\begin{equation}
   \bar T_1 = 8,19824546 N
\end{equation}

\begin{equation}
    \Delta T_1 = 0,380228328 N
\end{equation}

Logo, a tensão obtida pode ser expressa pelo resultado (38) :

\begin{equation}
    T_1 = (8,2 \pm 0,4) N
\end{equation}

Logo após isso, o grupo mediu a tração de ruptura pelo segundo método, medindo a massa necessária para romper o fio usando várias massas $M_{i}$ com auxílio das massas em forma de disco medidos com a balança de precisão descritos na tabela (2):

\begin{table}[!h]
 \centering
        \caption{Relação das massas medidas $M_{i}$}
         \begin{tabular}{| c | c |}
           \hline
             $i$  & $M_{i}[g](\pm0,1g)$  \\
            \hline
               1   &  907,0  \\
           \hline    
            2   &   895,9  \\
           \hline    
            3   &   859,7  \\
           \hline    
               4   &  825,6 \\
           \hline    
               5  & 792,6  \\
        
             
             \hline 
        \end{tabular} 
        \label{tab:2}

\end{table}

Assim, ao calcular-se a tração obteve-se (39):
\begin{equation}
    \bar T_2 = 8,3989296 
\end{equation}
E sua incerteza, determinada pelo desvio médio (40) :
\begin{equation}
    \Delta T_2 = 0,593207717
\end{equation}
De tal forma que pôde-se representar esse resultado por (41):
\begin{equation}
    T_2 = (8,4 \pm 0,6) N
\end{equation}
Finalmente, foi feita a equivalência entre essas duas medidas para encontrar (42):
\begin{equation}
|8,19824546 - 8,3989296| < 2(0,380228328 + 0,593207717)  \Rightarrow
0,20068414 < 1,94687208
\end{equation}
De fato são equivalentes
\subsection{Experimento 3}
Para a face lisa,construiu-se a seguinte tabela (3) para os ângulos obtidos, em roll: 


\begin{table}[!h]
 \centering
        \caption{Relação entre o ângulo $\theta$ obtido e o número $i$ da medição}
         \begin{tabular}{| c | c | c | c |}
           \hline
             $i$  & $\theta$($\pm 1^{o}$) & $i$  & $\theta$($\pm 1^{o}$)\\
            \hline
               1   &   10° &  11 & 12°   \\
           \hline    
            2   &   10°     & 12 &  12° \\
           \hline    
            3   &   10°  &   13 & 12°\\
           \hline    
               4   &  10° &   14 & 12° \\
           \hline    
               5  &   11°  &  15 & 13°\\
           \hline    
               6   &   11°  & 16 & 13°\\
           \hline    
               7    &  11°   &17 & 13°\\
           \hline
               8    &  11° & 18 & 13°\\
           \hline
               9  & 11° & 19 & 15° \\
            \hline 
             10 & 12°  &  20 & 15°\\
             
             \hline 
        \end{tabular} 
        \label{tab:2}

\end{table}
Após isso foi utilizada a relação (13), para obter todos os coeficientes de atrito estático e fez-se uma média, obtendo-se (43) :
\begin{equation}
\bar\mu_{e1} = 0,2118222
\end{equation}
E foi feito o desvio médio, que representa sua incerteza através do uso da equação (51), de forma a encontrar (44) :
\begin{equation}
\Delta\mu_{e1} = 0,021966735
\end{equation}
Assim, para face lisa,pode-se representar a medida por (45):
\begin{equation}
    \mu_{e1} = (0,21 \pm 0,02) 
\end{equation}
Por sua vez para fase rugosa, construiu-se a seguinte tabela (4), organizada em roll:
\begin{table}[!h]
 \centering
        \caption{Relação entre o ângulo $\theta$ obtido e o número $i$ da medição}
         \begin{tabular}{| c | c | c | c |}
           \hline
             $i$  & $\theta$($\pm 1^{o}$) & $i$  & $\theta $($\pm 1^{o}$)\\
            \hline
               1   &  15° &  11 & 17°   \\
           \hline    
            2   &   16°     & 12 &  17° \\
           \hline    
            3   &   16°  &   13 & 17°\\
           \hline    
               4   &   16° &   14 & 18° \\
           \hline    
               5  &   16°  &  15 & 18°\\
           \hline    
               6   &   17°  & 16 & 18°\\
           \hline    
               7    &  17°   &17 & 18°\\
           \hline
               8    &  17° & 18 & 19°\\
           \hline
               9  & 17° & 19 & 19° \\
            \hline 
             10 & 17°  &  20& 19°\\
             
             \hline 
        \end{tabular} 
        \label{tab:2}

\end{table}
\newpage
Após isso foi utilizada a relação (13), para obter todos os coeficientes de atrito estático e fez-se uma média, obtendo-se (46) :
\begin{equation}
\bar\mu_{e2} = 0,309671891
\end{equation}
E foi feito o desvio médio, que represneta sua incerteza através do uso da equação (51), de forma a encontrar (47) :
\begin{equation}
\Delta\mu_{e2} = 0,016495839
\end{equation}
Assim, para face rugosa,pode-se representar a medida por (48):
\begin{equation}
    \mu_{e2} =  (0,31 \pm 0,02) 
\end{equation}


\subsection{Discussão dos resultados}
No experimento 1 percebe-se que a equivalência funcionou para as massas, o que indica um valor condizente para as forças de tração calculadas no equilíbrio. Ademais, caso fossem consideradas forças de atrito entre as polias e as cordas ocorreria uma mudança nos resultados obtidos para as massas, em que dependendo do sentido estabelecido seriam obtidos valores distintos. Caso o atrito fosse uma força contrária ao peso, seria obtida uma massa maior do que o valor real, e caso fosse favorável ao peso seria obtida uma massa menor do que a esperada para o caso sem influência do atrito.

No caso do experimento 2, foi percebido que os valores para a tração da corda foram equivalente e bem próximos, o que leva a uma análise de que independente do método utilizado a tração deve dar o mesmo valor, se tratando de uma força que depende do fio.


Para o experimento 3 a conclusão feita, por meio da análise dos resultados, é que a face rugosa, por possuir uma superfície com maior pontos de contato com o plano inclinado, têm um maior coeficiente de atrito e portanto menor tendência a deslizar. 

\section{Conclusão}
Diante dos dados coletados, foi possível observar que os resultados obtidos nos experimento 1 mostraram-se equivalentes à medição direta das massas,fato que indica o experimento foi realizado com boa precisão. Também, no experimento 2, as duas maneiras de se medir a tração máxima suportada pelo fio também se mostraram equivalentes, o que implica que os dois métodos utilizados são igualmente precisos para esse propósito. No último experimento, mesmo diante de uma alta gama de valores de ângulos encontrados, a  incerteza da medição se mostrou baixa,indicando alta precisão, além disso, o resultado obtido foi conforme o esperado com a face rugosa tendo um maior coeficiente de atrito estático e com isso maior resistência ao deslizamento.
\section{Referências e Bibliografias}

\begin{thebibliography}{2}

\bibitem{Schneider-Laboratorio-Fisica} 
SCHNEIDER, J.F. \emph{Laboratório de Física I- Livros de Práticas.}


\bibitem{Tipler-Fisica-Engenheiros} 
TIPLER, P. A, Mosca, G. \emph{Física para cientistas e engenheiros.} 6.ed. Rio de Janeiro: Livros Técnicos Científicos, 2017.v.1.
\bibitem{Halliday-Fundamentos-Fisica} 
HALLIDAY, D.; RESNICK, R.; WALKER, J. \emph{Fundamentos de física.} 9.ed. Rio de Janeiro: LTC, 2012. v.l.
\end{thebibliography}
\section{Apêndice}
\subsection{Cálculos}
(Contas efetuadas pelo grupo de pesquisadores independentes)
\subsection{Conceitos secundários}
Foi utilizado para o valor tabelado da gravidade o encontrado em "Propriedades físicas" [3].
\begin{equation}
g = 9,81 m/s^2
\end{equation}
Tabém foram utilizadas as definições de média de medidas e desvio médio, encontradas em [1],resultando em (50) e (51):
\begin{equation}
    \bar{x} = \frac{1}{n}\sum_{i=1}^{n} x_i
\end{equation}
\begin{equation}
    \text{Desvio Médio} = \frac{1}{n}\sum_{i=1}^{n} |x_i - \bar{x}|
\end{equation}
Por fim, também foi utilizado a definição de equivalência dada no livro de práticas [3] (52):
\begin{equation}
    |x_1 - x_2| < 2(\sigma_1 + \sigma_2)
\end{equation}
\end{document}
