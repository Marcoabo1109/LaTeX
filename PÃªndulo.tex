\documentclass[12pt, letterpaper]{article}
\usepackage[utf8]{inputenc}
\usepackage[portuguese]{babel}
\usepackage{graphicx}
\usepackage{amsmath}
\usepackage{geometry}
\usepackage{caption}
\usepackage{pgfplots}
\usepackage{ulem}
\usepackage{circuitikz}
\usepackage{float}
\usepackage{indentfirst}
\usepackage{gensymb}
\usepackage{emoji}
\pgfplotsset{width=10cm, compat=1.18}
\geometry{
	paper=a4paper, 
	inner=3cm,
	outer=3cm,
	bindingoffset=.5cm, 
	top=2cm, 
	bottom=2cm, 
}


\graphicspath{ {./imagens/} }
\newcommand{\source}[1]{\caption*{Fonte: {#1}} }


\begin{document}

\begin{titlepage}
    \begin{center}
        \Large
       
        \vspace*{1cm}

        \includegraphics[scale=0.75]{imagens/ifsc_logo.png} \\
        \Huge
        \textbf{Universidade de São Paulo} \\
        \huge
        Movimento Unidimensional
        
        \Large
        Laboratório de física I

        \vspace{4cm}
        \Large
        
    \end{center}
   
    \begin{flushright}
        \Large
        \textbf{Alunos}\\
        \large
            Luis Eduardo Aires Coimbra - n° 15472565\\
            Marco Antônio Bicalho de Oliveria - n° 15474741\\
            Mateus Alves Martins Claudino - n° 15478791
            
    \end{flushright}

    \begin{flushright}
        \Large
        \textbf{Professor}\\
       Rafael Victorio Carvalho Guido
        \large
            
            
    \end{flushright}
    
    \vspace*{\fill}
    \centering \large São Carlos \\ 2024
    
\end{titlepage}
 \tableofcontents
 \newpage

\section{Introdução}
Um dos objetivos da física é estudar o movimento dos objetos: a rapidez com
que se movem, por exemplo, ou a distância que percorrem em um dado intervalo de
tempo. Dessa forma, pode-se especificar o movimento em uma única dimensão,o movimento nos casos em que o objeto está se movendo em linha reta. Este
tipo de movimento é chamado de movimento unidimensional[8].

De forma geral, se esse movimento for uniformimente variado ele pode ser descrito pela equação (1) [10]:
\begin{equation}
 x(t) = x_{0} + v_{0}t + \frac{at^2}{2}
\end{equation}
Com $x(t)$ representando a posição final do objeto em um instante genérico $t$, $x_{0}$ a posição inicial em $t = 0$, $v_{0}$ a velocidade inicial em $t = 0$, $a$ a aceleração, e $t$ o instante de tempo analisado, com $t \geq 0$.

    \subsection{Pêndulo}
        O pêndulo simples consiste em um fio de comprimento $L$ que possui uma de suas extremidades fixa e a outra presa a um corpo de massa $m$. No estudo do seu movimento, abandona-se o pêndulo de um ângulo inicial $\phi_{0}$ em relação à vertical, quando está sujeito a um campo gravitacional de aceleração $\vec g$, o que causa um movimento oscilatório descrito por um arco de circunferência e de período $T$. Para $\phi_{0}$ suficientemente pequeno, o movimento do pêndulo simples pode ser aproximado a um movimento harmônico simples (MHS) [10]. 
        
        Assim, ao se deduzir as equações de seu movimento com base em um MHS, obtém-se a equação (2) para o cálculo do seu período de oscilação.
        \begin{equation}
            T = 2\pi \sqrt{\frac{L}{g}}
        \end{equation}
        A equação (2) foi inicialmente descrita por Galileu Galilei, ao estudar a periodicidade de pêndulos simples, assim, para valores fixos de $L$, ao medir o período $T$ é possível estudar a aceleração da gravidade $\vec g$ no local do experimento.
        
    \subsection{Plano inclinado}
       Por plano inclinado, refere-se a um plano fixo que forma um ângulo $\theta$ com a horizontal, com uma massa deslizando sobre ele sem atrito. Existem duas forças atuando: o peso e a reação do plano. Como essas forças não se cancelam, existe uma força resultante e, portanto, aceleração.      
       O movimento pode ser descrito como unidimensional, escolhendo um referencial que meça a coordenada na direção paralela ao plano inclinado. Definindo o instante t = 0, como aquele cujo objeto está na posição $y = 0$, com velocidade $v_0$, a equação horária (3) resulta em [9]:
       \begin{equation}
          y(t) = v_{0}t + \frac{a}{2}t^2
       \end{equation}
       Sendo $a$ a aceleração dada pela equação (4) :
       \begin{equation}
          a = g\sin{\theta}
       \end{equation}
       
    \subsection{Regressão linear}
 O Método dos Mínimos Quadrados (MMQ), ou Mínimos Quadrados Ordinários (MQO) ou OLS (do inglês Ordinary Least Squares) é uma técnica de otimização matemática que procura encontrar o melhor ajuste para um conjunto de dados tentando minimizar a soma dos quadrados das diferenças entre o valor estimado e os dados observados (tais diferenças são chamadas resíduos) [9].

Esse é o método analítico geral para encontrar a melhor reta
que represente o conjunto de N pares de dados experimentais
 yixi , com i = 1, ...N, independente de critérios do observador. A
ideia fundamental é definir a melhor reta como aquela que minimiza
as distâncias verticais em relação aos dados experimentais. O
método de mínimos quadrados, ou regressão linear, considera a
soma dos quadrados das distâncias (5)[9]:

\begin{equation}
    S= \sum_{i=1}^{N}(y_{Ci}-y_{i})^{2}
\end{equation}

em que $y_{Ci}$ é o valor calculado para o dado i-ésimo com a equação
da melhor reta $y_{Ci}= ax+b$ . O processo de minimização de S
,como função dos parâmetros da reta, fornece as seguintes
expressões (6)(7)(8)(9)(10):

\begin{equation}
    a= \frac{\sum(x_{i}-\bar x)y_{i} }{\sum(x_{i}-\bar x)^2}
\end{equation}

\begin{equation}
    b = \bar y - a \bar x
\end{equation}

\begin{equation}
    \Delta y = \sqrt{\frac{\sum(ax_{i} +b - y_{i})^2 }{N-2}}
\end{equation}

\begin{equation}
    \Delta a = \frac{\Delta y}{\sqrt{\sum(x_{i}-\bar x)^2}}
\end{equation}

\begin{equation}
    \Delta b = \sqrt{\frac{\sum x_{i}^2 }{N\sum(x_{i}-\bar x)^2}}\Delta y
\end{equation}

    Sendo $a$ o coeficiente angular da reta, $b$ o coeficiente linear, $\Delta y$ o erro de $y$, $\Delta a$ o erro de $a$ e $\Delta b$ o erro de $b$.

\section{Objetivos}
Este estudo visou aprofundar a compreensão do movimento unidimensional através de experimentos que exploram a determinação da gravidade usando o pêndulo simples e o plano inclinado. Além disso, buscou-se familiarizar os estudantes com o conceito de linearização de funções pelo método dos mínimos quadrados. Os objetivos específicos incluiram proporcionar uma compreensão sólida do movimento unidimensional, permitindo aos estudantes calcular experimentalmente a aceleração devida à gravidade, bem como introduzi-los à técnica de linearização de dados, promovendo, assim, o desenvolvimento de habilidades práticas e conceituais em física.



\section{Materiais}
    \begin{enumerate} 
        \item Massa com suporte fixo 
        \item Régua plana de plástico
        \item Cronômetro Casio-hs-6-10-MRKEAN
        \item Trena métrica - Starrett
        \item Trilho de ar
        \item Fita de papel termosensível
        \item Eletroimã
        \item Carrinho
        \item Agulha
        \item Barbante
    \end{enumerate}
\section{Metodologia}
    \subsection{Experimento do pêndulo}
    Nesse experimento o grupo utilizou-se de um pêndulo para determinar a aceleração da gravidade local a partir da medição do seu período.

    
   Na execução do procedimento, primeiramente, mediu-se o comprimento do barbante com uma trena, seguido pela cronometragem de dez períodos de oscilação do pêndulo, com o objetivo de minimizar o erro de medição para apenas um ou dois períodos. Os resultados obtidos foram então registrados na tabela 1. Então, calculou-se o período de uma só oscilação dividindo o tempo encontrado por dez, e em seguida elevou-se esse período ao quadrado, com os resultados subsequentemente inseridos na tabela 1. Esse processo foi repetido por um total de sete vezes.

    
    Para a análise da relação do período de oscilação $T$ do pêndulo com o comprimento $L$ do barbante a que ele está preso, foi feita a linearização da equação (2), obtendo o seguinte resultado (11):
    \begin{equation}
    T^2 = \frac{(2\pi)^2}{g}L 
    \end{equation}
    Ou ainda (12) :
    \begin{equation}
    T^2 = kL
    \end{equation}
    Sendo k o coeficiente angular da reta de melhor ajuste de $T^2$ por $L$.

    
   A partir dos dados da tabela, foi elaborado um gráfico milimetrado de $T^2 \times L$. Utilizando-se o método dos mínimos quadrados (MMQ), foi determinada a inclinação da reta, denotada por $k$, junto de sua incerteza, calculada pelas equações (8) e (9). Posteriormente, igualou-se o valor de $k$ ao fator $\frac{(2\pi)^2}{g}$, resultando na obtenção da aceleração da gravidade, conforme evidenciado na equação (13):
    \begin{equation}
     g = \frac{(2\pi)^2}{k}
    \end{equation}
    E também sua incerteza (14) :
    \begin{equation}
    \Delta g = \frac{(2\pi)^2\Delta k}{k^2}
    \end{equation}
    \subsection{Experimento do plano inclinado}
    Para o experimento 2, foi utilizado o conceito do plano inclinado. Apoiou-se um objeto num trilho de ar, para diminuir a influência do atrito, e após isso inclinando-o de um ângulo $\theta$ com a ajuda de um calço de madeira. 
    
    O ângulo foi encontrado através do seguinte triângulo retângulo formado pelo plano inclinado:
    (desenho do triangulo retangulo).

    
    a altura desse triangulo foi medida com um paquímetro e a hipotenusa com uma trena. Assim, obteve-se o ângulo $\theta$ pela equação (15):
    \begin{equation}
        \theta = \arcsin{\frac{D}{H}}
    \end{equation}
    
    Com o auxílio de um eletroímã, um oscilador de frequência $f$, e uma fita termossensível, foi marcada a posição do objeto no plano inclinado a cada intervalo de tempo $\Delta t$ constante dado pela equação (16): 
    \begin{equation}
        \Delta t = \frac{1}{f}
    \end{equation}
    Em seguida, foi medida a posição $y$ de cada ponto marcado na fita termossensível. Após isso, os dados foram colocados na tabela (2) para melhor análise.

    Rearranjando a equação (3), foi obtida a equação (17):
    \begin{equation}
        \frac{y(t)}{t} = v_{0} + \frac{at}{2}
    \end{equation}
    Assim, foi plotado um gráfico em escala milimetrada de $\frac{y}{t} \times t$, que aproxima-se de uma reta com coeficiente angular dado por $\frac{a}{2}$, sendo $a$ descrito pela equação (4) e o coeficiente angular calculado pelo método dos mínimos quadrados.
\newpage
\section{Resultados}
    \subsection{Resultados do experimento do pêndulo}
    Após aferir os períodos $T$ em função dos comprimentos $L$ foi obtida a seguite tabela:

    
    \begin{table}[!h]
    \centering
    \caption{Comprimento do pêndulo $L$, tempo de N oscilações $t_{N}$, período de
oscilação $T$ e valores de $T^2$ para linearização dos dados}
     \begin{tabular}{| c | c | c | c |}
   \hline
     $L$(m)($\pm 0,001$ m)  & $t_{10}$(s)($\pm 0,2$ s) &  $T$(s)($\pm 0,02$ s) & $T^2$($s^2$)   \\
    \hline
       2,362   &   30,6   &   3,06 & 9,3 ($\pm 0,1$)     \\
   \hline    
       2,100   &   28,8  &   2,88  & 8,3 ($\pm 0,1$)    \\
   \hline    
       1,810   &   27,0  &   2,70  & 7,3 ($\pm 0,1$)   \\
   \hline    
       1,532   &   24,8  &   2,48  & 6,1 ($\pm 0,1$) \\
   \hline    
       1,260   &   22,6  &   2,26  & 5,09 ($\pm 0,09$) \\
   \hline    
        0,940   &   19,3  &   1,93  & 3,73 ($\pm 0,08$) \\
   \hline    
        0,652    &  16,1  &   1,61  & 2,59 ($\pm 0,06$) \\
   \hline
\end{tabular} 
    \label{tab:1}
    \end{table}

    
 Em seguida,foi desenhado o gráfico, que, conforme o indicado pela equação (11), possui uma relação linear:


\begin{figure}[!h]
    \centering
    \caption{Gráfico em escala milimetrada de $T^2 \times L$}
        \begin{tikzpicture} 
        
        \begin{axis}[    
        xlabel=L(m),            
        ylabel= $T^2(s^2)$]
        
        \addplot[only marks, mark=*] coordinates {
            (2.362 , 9.3)
            (2.100 , 8.3)
            (1.810 , 7.3)
            (1.532 , 6.1)
            (1.260 , 5.09)
            (0.940 , 3.73)
            (0.652 , 2.59)
        };
        
        \addplot[domain=0:2.5] {3.94886*x};
        
        \end{axis}
        \end{tikzpicture}
    \label{Elaborado pelos autores}
\end{figure}



Posteriomente, utilizou-se do método dos mínimos múltiplos quadrados (MMQ), explicitado na equação (6), para obter o coeficiente angular da reta, denotado por k (18):
\begin{equation}
 k = 3,94886 \hspace{0.1cm} s^2/m
 \end{equation}
 E também sua incerteza, por meio das equações (8)  e (9), com o seguinte resultado (19):
 \begin{equation}
 \Delta k = 0,037389256  \hspace{0.1cm} s^2/m 
 \end{equation}
 De tal modo que esse coeficiente pode ser representado por (20) :
 \begin{equation}
    k = (3,95 \pm 0,04) \hspace{0.1cm} s^2/m 
 \end{equation}
 A partir disso foi possível substituir $k$ na equação (13) para obter a  aceleração da gravidade (21) :
 \begin{equation}
    g = 9,99741\hspace{0.1cm} m/s^2
 \end{equation}
 Além disso foi calculada sua incerteza, substituindo os resultados das equações (19) e (20) na equação (14), para obter o resultado representado na equação (22) :
\begin{equation}
  \Delta g = 0,094659093 \hspace{0.1cm} m/s^2
\end{equation}
De tal modo que é possível representar esse resultado pela equação (23) :
    \begin{equation}
    g = (10,00 \pm 0,09) \hspace{0.1cm} m/s^2
    \end{equation}
    Por último, foi feito um cálculo da equivalência entre o valor de
$g$ encontrado e o tabelado, de tal modo que obteve-se (24) :
    \begin{equation}
     |9,99741 - 9,81| < 2|0,094465093|\hspace{1.5cm}    0,18741 < 0,188930186
    \end{equation}
    De fato, são equivalentes.
    \subsection{Resultados do experimento do plano inclinado}
    Inicialmente, foram aferidos os valores de $H$ e $D$ mostrados na equação (15), dados pelas equações (25), (26):
    \begin{equation}
        H = 1,600 \pm 0,001 m
    \end{equation}
    \begin{equation}
        D = 0,06100 \pm 0,00005 m
    \end{equation}
    Em seguida, foi calculado o valor de $\theta$ através da equação (15), dado pela equação (27):
    \begin{equation}
        \theta = 2,184931117 \degree \Rightarrow \sin{\theta} = 0,038124999
    \end{equation}
    Paralelamente, foi utilizada a equação (15) como base para calcular a incerteza do $\sin{\theta}$, assim foi gerada a equação (28):
    \begin{equation}
        \Delta \sin{\theta} = 0,000055078125
    \end{equation}
    Após truncamento e arredondamento, foi obtida a equação (29):
    \begin{equation}
        \sin{\theta} = 0,03812 \pm 0,00006
    \end{equation}
    Após isso, com a frequência $f = 5Hz$ do oscilador, usando a equação (16) foi obtida a equação (30) o intervalo de tempo entre cada marcação de posição na fita termossensível:
    \begin{equation}
        \Delta t = 0,2 s
    \end{equation}
     Então, com os valores  da posição $y$ em função do tempo, foi feita a seguinte tabela:

     
    \begin{table}[!h]
        \centering
        \caption{relação entre tempo $t$ após o início do movimento e a posição $y$ do carrinho paralela ao plano inclinado}
         \begin{tabular}{| c | c | c | c |}
           \hline
             $t$(s)  & $y$(m)($\pm 0,001$ m) &  $\frac{y}{t}$(m/s)($\pm 0,001$ m/s)\\
            \hline
               0,4   &   0,0290   &   0,073   \\
           \hline    
               0,6   &   0,064   &   0,107   \\
           \hline    
               0,8   &   0,115  &   0,144   \\
           \hline    
               1,0   &   0,182  &   0,182   \\
           \hline    
               1,2   &   0,263  &   0,219 \\
           \hline    
               1,4   &   0,359  &   0,256   \\
           \hline    
               1,6    &  0,471  &   0,294  \\
           \hline
               1,8    &  0,598  &   0,332  \\
           \hline
        \end{tabular} 
        \label{tab:2}
    \end{table}

Em seguida, foi plotado o gráfico (2) em escala milimetrada para visualizar a relação entre as grandezas, que no papel milimetrado deve ser linear com base na equação (17).

\begin{figure}[!h]
    \centering
    \caption{Gráfico em escala milimetrada de $\frac{y}{t} \times t$}
        \begin{tikzpicture}
        \begin{axis}[    
        xlabel= t(s),            
        ylabel= y/t (cm/s) ]
         
        \addplot[only marks, mark=*] coordinates {
            (0.4 , 0.0725 )
            (0.6 , 0.1067 )
            (0.8 , 0.1438)
            (1.0 , 0.1820)
            (1.2 , 0.2192)
            (1.4 , 0.2565)
            (1.6 , 0.2944)
            (1.8 , 0.3322)
        };
        
        \addplot[domain=0:2.5] {0.186410714*x - 0.0041392854};
             
        \end{axis}
         
        \end{tikzpicture}
    \label{fig:Elaborado pelos autores}
\end{figure}

    


    Com a ajuda do método dos mínimos quadrados, através da equação (6) e o $\bar t$, foi calculado o coeficiente angular da reta do gráfico (2), previsto pela equação (17), assim sendo obtido o resultado da equação (32):
    \begin{equation}
        \bar t = 1,1 s
    \end{equation}
    \begin{equation}
        \frac{a}{2} = 0,186410714 m/s^2
    \end{equation}
    Também foi calculado o $\bar {\frac{y}{t}}$, e assim obteve-se o valor do coeficiente linear pela equação (7), da incerteza de $\frac{y}{t}$ pela equação (8) e da incerteza dos coeficientes pelas equações (9) e (10); que foram referenciados respectivamente pelas equações (33), (34), (35), (36), (37):
    \begin{equation}
        \bar {\frac{y}{t}} = 0,2009125 m/s
    \end{equation}
    \begin{equation}
        b = -0,0041392854 m/s
    \end{equation}
    \begin{equation}
        \Delta \frac{y}{t} = 0,001143277451 m/s 
    \end{equation}
    \begin{equation}
        \Delta \frac{a}{2} = 0,000882057703 m/s^2 
    \end{equation}
    \begin{equation}
        \Delta b = 0,0010510930621 m/s 
    \end{equation}
    Após feitos os devidos truncamentos e arredondamentos foi obtida a equação (38): 
    \begin{equation}
        \frac{a}{2} = 0,1864 \pm 0,0009 m/s^2
    \end{equation}
    Depois de achados os coeficientes, utilizando a equação (4) foi encontrado o valor da gravidade, expressado pela equação (39), e também foi calculado o valor de sua incerteza expressado na equação (40):
    \begin{equation}
        g = \frac{a }{\sin(\theta)} \Rightarrow g = 9,778922703 m/s^2
    \end{equation}
    \begin{equation}
        \Delta g = 0,060399218 m/s^2
    \end{equation}
    Assim, após truncar os valores foi gerada a equação (41):
    \begin{equation}
        g = 9,78 \pm 0,06 m/s^2
    \end{equation}
    Por último, foi feito um cálculo da equivalência, equações (42) e (43), entre o valor de $g$ encontrado no experimento 2 com o valor do experimento 1, dado pela equação (39), e com o valor de referência da gravidade:
    \begin{equation}
        |9,778922703 - 9,99741| < 2\cdot |0,060399218 + 0,094659093| \Rightarrow 0,218487297 < 0,310116622
    \end{equation}
    \begin{equation}
        |9,778922703 - 9,81| < 2\cdot |0,060399218| \Rightarrow 0,031077297 < 0,120798436
    \end{equation}
    Portanto, ambas as gravidades são equivalente ao valor encontrado no experimento 2.
    \subsection{Discussão dos resultados}
    Para estimar qual dos dois experimentos resulta num melhor valor para a gravidade , foi feito um cálculo de desvio de cada valor encontrado em comparação com o valor esperado dado pelo valor tabelado da gravidade utilizando a fórmula (44) seguinte:
    \begin{equation}
        \sigma_{erro} = \frac{|Valor_{calculado} - Valor_{esperado}|}{Valor_{esperado}}
    \end{equation}
    Para o primeiro experimento:
    \[
        \sigma_{1} = \frac{|9,99741 - 9,81|}{9,81} = 0,019103975
    \]
    Para o segundo experimento:
    \[
        \sigma_{2} = \frac{|9,78 - 9,81|}{9,81} = 0,003167920183
    \]
    Portanto, a gravidade calculada no segundo experimento é mais adequada, pois seu desvio em relação ao valor esperado é inferior ao do primeiro($\sigma_2 < \sigma_1$), além de que sua incerteza também é inferior a da gravidade do experimento 1.

    Além disso, no Experimento 2, foi determinado o valor do coeficiente linear na equação (34), que representa, segundo a equação (17), a velocidade inicial do carrinho no plano inclinado. Teoricamente, essa velocidade inicial deveria ser nula. No entanto, devido a pequenas imprecisões causadas pela limitação dos equipamentos de medição, um valor residual foi observado. Contudo, dada a ordem de grandeza deste resultado, pode-se considerá-lo insignificante. Tal constatação reforça a validade do experimento e dos resultados obtidos em relação à gravidade.

\section{Conclusão}

Diante dos dados coletados, foi possível observar que o experimento 2, o do plano inclinado, demonstrou maior precisão em comparação ao experimento 1. Isso se deve ao fato de que, no experimento 1, os erros sistemáticos eram mais prováveis devido às múltiplas medições e testes realizados. A realização de um maior número de medições, embora possa aumentar a precisão das medidas individuais, também pode ampliar a margem para erros sistemáticos. No entanto, no experimento 2, as condições controladas proporcionaram uma redução significativa na incidência de erros sistemáticos, resultando em dados mais confiáveis e precisos. 

\section{Apêndice}
\subsection{Cálculos}
 (Contas efetuadas pelo grupo de pesquisadores independentes)
\subsection{Conceitos Secundários}

\section{Referências e Bibliografias}

\begin{thebibliography}{9}
  
\bibitem{ABNT-NBR7195} 
ABNT NBR 7195

\bibitem{ACI-318-14} 
ACI 318-14

\bibitem{ASTM} 
ASTM

\bibitem{ASTM-C597-09} 
ASTM C597-09

\bibitem{ASTM-E1876-09} 
ASTM E1876-09

\bibitem{Eurocode-2004} 
Eurocode 2004

\bibitem{fib-Model-Code-2010} 
fib Model Code 2010

\bibitem{Halliday-Fundamentos-Fisica} 
HALLIDAY, D.; RESNICK, R.; WALKER, J. \emph{Fundamentos de física.} 9.ed. Rio de Janeiro: LTC, 2012. v.l.

\bibitem{Schneider-Laboratorio-Fisica} 
SCHNEIDER, J.F. \emph{Laboratório de Física I- Livros de Práticas.}

\bibitem{Tipler-Fisica-Engenheiros} 
TIPLER, P. A, Mosca, G. \emph{Física para cientistas e engenheiros.} 6.ed. Rio de Janeiro: Livros Técnicos Científicos, 2017.v.1.



\end{thebibliography}
\end{document}
