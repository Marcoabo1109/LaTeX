\documentclass{article}
\usepackage{graphicx} % coloca imagens
\usepackage{amsmath} % coloca símbolos de matemática
\usepackage{multicol} % mais de uma coluna, tipo jornal
\usepackage{indentfirst} 
\usepackage{geometry}
\title {Redação}
\author{ Modelo ITA}
\date{2023}

\begin{document}
\maketitle
\newpage


\section*{Sumário}
\subsection*{Modelo ............................................................................................... 3}
\hspace{1,5in}

\subsection*{Redações Semanais ............................................................................ 4}
\hspace{1,5in}

\subsection*{Redações em Vestibulares ................................................................. 14 } 
\hspace{1,5in}

\subsection*{Repertórios ....................................................................................... 20}

\newpage
\section*{Indrodução}
\noindent
Contextualização + Ponte + Tese ( envolto dos Tópicos Frasais [T.F]) $\rightarrow$ Desdobramento.


•   $1^o$ período (de 3 a 4 linhas):  Coesivo de conformidade ( segundo, consoante, de acordo...) + Aposto entre vírgulas relevante para o tema sobre a autoridade + conceito por paráfrase. [caso não seja uma autoridade, um livro, uma série, repertório externo].



• $2^o$ período (de 3 a 4 linhas): Ponte com todas as palavras da frase tema, já direcionada para a tese - criando uma ponte que seja estratégica a partir das palavras que relacionam os conceitos com o tema.


• $3^o/4^o$ período ( de 3 a 4 linhas): Coesivo de conclusão ou continuidade + Tese resposta + T.F de A1 + T.F de A2.
\section*{Desenvolvimento}
 • $1^o$ período (de 1 a 2 linhas): Coesivo + Tópico frasal - colocando, se possível, um conceito ou  que esteja diretamente ligado à autoridade a ser citada em seguidada, gerando um campo semântico.


• $2^o$ período (de 3 a 4 linhas): Coesivo de conformidade + nome da autoriadade com pequenos adornos + aposto entre vírgulas relevante para o tema sobre a autoridade + conceito exposto por paráfrase.


• $3^o/4^o$ período (de 1 ou 2 linhas): Ponte com o tema  - premissa intermediária que mostra a relação do conceito com o tema    ou 
             Coesivo de exemplificação + Evidência que confirme o conceito - atuando como                      comprobatório do teórico.


• $4^o/5^o$ período (de 2 a 3 linhas): Análise da evidência de forma argumentativa.


• $5^o/6^o$ período (de 1 a 2 linhas): Coesivo de conclusão ou continuidade + Tese resposta (retomada do tema e conclusão lógica do T.F) + Desdobramento como abertura para A2. 

\section*{Conclusão}
• $1^o$ período (de 1 a 2 linhas): Conectivo de conclusão deslocado por vírgulas + Retomada afirmativa da tese já provada + breve resumo a partir do T.F 1 e 2.


• $2^o$ período (de 1 a 2 linhas): Conectivo de 
sequenciamento + Prognóstico de acordo com o tema.

\newpage
\section*{Redações Semanais} \paragraph{Tema:} A construção da verdade no século $XXI$

\begin{flushright} Marco Antônio \end{flushright}
Pós-Verdade: O Poder da Narrativa na Sociedade do Discurso


Segundo Michel Foucault, na obra “A ordem do discurso", a verdade tende a ser utilizada de modo a controlar e regular a sociedade. Ao estabelecer a relação entre poder e saber, Foucault propõe uma nova definição que permite que a força do discurso atue de forma negativa, distorcendo a verdade e garantindo a dominação do poder opressor. Ele argumenta que essa forma de “ameaça" ocorre por meio da manipulação do saber. Portanto, a construção da verdade na sociedade contemporânea depende daqueles que detêm o poder sobre as informações e suas formas de difusão, e não necessariamente de um fundamento axiomático sólido e bem elaborado.

Em primeira análise, em uma sociedade inserida em um ambiente digitalizado, o poder informacional não está limitado ao acesso de informações, mas sim à maneira como estas são difundidas e propagadas. Conforme Foucault, a sociedade se disciplina através da linguagem das ideias que se proliferam indefinidamente, caracterizando a Sociedade do Discurso. Dessa forma, é notório como o discurso esteve impregnado na sociedade, sendo importante para a formação de grandes territórios organizados politicamente, como as pólis gregas - que utilizavam as ágoras públicas para a difusão de ideias. No século XXI, o tecido social ampliou-se, pois os moldes de proliferação de ideias e discursos foram transfigurados com o advento da internet, que se tornou uma ágora digital. Devido a esse modo de disseminação rápida e indiscriminada, o poder torna-se mascarado e dificulta a percepção do que realmente é verdade ou apenas uma dialética sofista e erística. Ao mesmo tempo em que se camuflam, os ideais se perpetuam e influenciam em grande escala o comportamento do homem social. Isso pode ser evidenciado nos resultados das eleições de 2016 nos EUA e de 2018 no Brasil, em que as fake news foram utilizadas como recurso de manipulação de massa, contribuindo para colocar no poder aqueles que detiveram o controle das informações - mesmo que estas fossem erísticas e falaciosas. Portanto, o poder é estabelecido no controle da propagação das informações.

Por conseguinte, na sociedade do discurso, a verdade factual e a evidência objetiva são, muitas vezes, consideradas menos relevantes do que as narrativas persuasivas e emocionalmente cativantes. Esse contexto esclarece o conceito de pós-verdade, complementando a teoria do filósofo em sua obra “Microfísica do poder", que retoma a relação entre poder e saber na contemporaneidade com o objetivo de produzir “verdades" cujo interesse essencial é a dominação do homem. Dessa forma, mesmo ao tentar combater ideias infundadas, como ocorreu nas recentes eleições - como afirmar que Barack Obama fundou o Estado Islâmico, no caso de Trump, e que o comunismo dominaria o Brasil, no caso de Bolsonaro - denunciar essas informações como falsas não foi suficiente para mudar o voto majoritário. Diante desses fatos, a proposta de instituir a PL 2630 no Brasil busca exatamente a regulamentação das informações e a taxação das “Big Techs" no contexto da propagação de informações falsas. Isso reforça a ideia do poder das mídias na circulação e construção de informações na internet, já que são capazes de disseminar discursos mais cativantes e persuasivos, mesmo que não sejam embasados em fatos concretos.

A construção da verdade no mundo contemporâneo depende, portanto, daqueles que detêm o poder de persuasão no discurso e da sua rápida difusão, independentemente de a narrativa ser erística ou não. Como prognóstico, prevê-se, a fortiori, a perpetuação da pós-verdade na atual sociedade do discurso.
\newpage
\section*{comentários}
\begin{enumerate}
\item Tema: Ideias de textos da coletânea identificadas e utilizadas a partir de uma leitura crítico-interpretativa. 1,5|2
\item Tipo de texto: Raciocínio completo e que revela planejamento do texto por apresentar certa complexidade, mas com pontual quebra, salto ou lacuna que não compromete a progressão textual e a defesa da tese. 1,5|2
\item Coerência: Há ocorrência pontual de informação relevante para compreensão do tema e/ou do argumento não suficientemente desenvolvida. 1,5|2
\item Coesão: Além de estratégica, a coesão acompanha a complexidade e sofisticação do raciocínio com êxito. Parte desse êxito decorre do uso variado, autônomo e preciso dos recursos de coesão, não havendo mecanicidade ou ausência de conector necessário à formação de sentido. 2|2
\item Modalidade: Poucos desvios, não prejudicam a leitura e compreensão das ideias. 1,5|2



\end{enumerate}
\begin{figure}
\centering
\includegraphics[width=0.48\linewidth]{CamScanner 06-19-2023 13.54_1.jpg}
\caption{\label{}A redação possui 574 palavras, e na folha de redação foram utilizadas 33 linhas}
\end{figure}

\newpage
\paragraph{Tema:} Produções culturais, poder e criticidade

\begin{flushright} Marco Antônio \end{flushright}
 Hipermodernidade: o poder do discurso nas produções culturais


Segundo Michel Foucault, na obra “A ordem do discurso", o discurso tende a ser utilizado de modo a controlar e regular a sociedade. Ao estabelecer a relação entre poder e discurso, Foucault propõe uma nova definição que permite que a força deste atue de forma negativa, distorcendo a verdade e garantindo a dominação do poder opressor. Ele argumenta que essa forma de "ameaça" ocorre por meio da manipulação do saber e, consequentemente, pela prevalência da soberania nas produções culturais. Nesse sentido, a formação de um pensamento crítico, analisado nas atuais produções culturais (artísticas, como filmes e músicas), é prejudicada em uma sociedade cuja lógica de mercado rege todo o tecido social. Dessa forma, o poder é atribuído àqueles que detêm os meios para difundir e seduzir os indivíduos ao consumo de suas produções culturais.

Em primeira análise, a criticidade é retardada em uma sociedade neoliberal pautada na máxima do lucro, pois as produções culturais são influenciadas pelo poder vigente do capitalismo. Nesse prisma, conforme o sociólogo Gilles Lipovetsky, em sua obra “Era do Vazio", a sociedade hipermoderna é caracterizada pelas constantes transfigurações das relações humanas e sociais em mercadoria. Tal teoria se aplica nas atuais produções cinematográficas, nas quais a relação de mercadoria se estabelece não apenas no fato da própria arte se materializar em objeto de mercado, mas também em toda a movimentação mercantil em paralelo à produção, aturdindo a possível criticidade da obra e relegando-a a um segundo plano. Isso é perceptível no recente lançamento do filme Barbie, que não somente angariou milhões com a própria obra, mas também mobilizou milhares de compras associadas ao marketing difundido com a cor rosa, correlacionada à estereotipização dos itens utilizados pela protagonista. Assim, o poder da lógica neoliberal é camuflado de forma sutil, levando os indivíduos a convergirem para um consumo cada vez mais inconsciente, e o pensamento crítico é, portanto, ofuscado pela influência do consumo.

Além disso, o poder é atribuído à produção cultural à medida que ela se torna uma ferramenta de manipulação para quem detém os meios de sua difusão. Conforme Foucault observou em seus estudos, a sociedade contemporânea se submete a um processo de disciplinação que se desenrola principalmente por meio da sedução da linguagem das ideias, que se proliferam indefinidamente, caracterizando a Sociedade do Discurso. Nesse contexto, aqueles que possuem meios de difusão mais abrangentes, como os veículos de mídia de massa e as plataformas digitais, desempenham um papel crucial na disseminação e legitimação de narrativas culturais. A título de exemplo, temos as produções cinematográficas e musicais dos EUA, que compõem mais de 50 por cento do que é consumido no mundo nas maiores plataformas digitais, como You Tube, Netflix e Spotify, de acordo dados da BBC news. Dessa forma, a influência cultural está intimamente ligada ao exercício do poder, formando uma espécie de microfísica do poder inserida nas produções culturais, sejam elas críticas ou não. Essa dinâmica não apenas molda as percepções e valores da sociedade, mas também afeta diretamente a construção do conhecimento e a formação das identidades individuais e coletivas, influenciando, assim, a criticidade dos sujeitos. Logo, a influência das produções culturais não é apenas uma questão de disseminação de informações, mas também está intrinsecamente ligada ao exercício do poder.

Portanto, a criticidade é ofuscada perante à lógica neoliberal de consumo, tornado as produções culturais ferramentas de poder e dominação para aqueles que detêm a maioria dos meios de difusão das narrativas culturais. Assim, o poder do discurso camufla-se em meio às produções sociais e corrobora a manutenção de uma hipermodernidade.
\newpage
\begin{figure}
\centering
\includegraphics[width=0.48\linewidth]{CamScanner 09-06-2023 10.58_1.jpg}
\caption{\label{}A redação possui 588 palavras, e na folha de redação foram utilizadas 35 linhas}
\end{figure}
\section*{comentários}

\begin{enumerate}
\item Tema:Ideias de textos da coletânea identificadas e utilizadas de maneira produtiva ou autoral, a partir da leitura crítico-interpretativa da coletânea. Uso pertinente e produtivo de repertório legitimado e articulado plenamente às explicações propostas na comprovação das ideias centrais 2|2
\item Tipo de texto: Raciocínio completo e que revela planejamento do texto por apresentar certa complexidade, mas com pontual quebra, salto ou lacuna.1,5|2
\item Coerência: Compreensão plena dos conceitos-chave. Não há incoerência externa, mas pode haver pontuais incoerências internas que não comprometem a consistência da abordagem temática. 1,5|2
\item Coesão: Há continuidade bem delimitada. Entretanto, a construção do raciocínio ainda não é suficientemente complexa e estratégica para que os conectivos sejam mobilizados de modo a contribuir efetivamente para o êxito desse tipo de raciocínio. 1,5|2
\item Modalidade: Bom domínio da norma-padrão: há poucos desvios, que não prejudicam a leitura. 1,5|2
\end{enumerate}

\newpage

\paragraph{Tema:} É possível haver lucro sem exploração ?

\begin{flushright} Ivan Cunha \end{flushright}
O dolar vale mais do que o sangue 

Exploração ocorre quando um indivíduo se beneficia desproporcionalmente da sua relação com outrem. Essa definição é reminiscente das relações trabalhistas contemporâneas justamente pela dificuldade em igualar o quanto os chefes e os operários se beneficiam com os trabalhos destes. Nesse sentido, mesmo com a possibilidade teórica de se estabelecer uma relação equilibrada de trabalho, uma vez que se prioriza o ganho econômico ao trabalhador, a referida teoria se torna ficção. Com efeito, as empresas exploram os funcionários para obter lucro em sociedades em qualquer grau de desemvolvimento.

Primeiramente, a falta de recurso é usada como ferramenta de exploração. A princípio, em sociedades pobres a movimentação do mercado é menos ativa. Por isso, empresas pequenas e médias buscam mercados mais ativos por esperar deles maior chance de lucro. No entanto, essa vacuidade de competição facilita a formação de monopólios pelas multinacionais com recurssos para fazê-lo. Assim, com o poder de estabelecer sozinhas o preço dos produtos e o valor dos salários, as grandes empresas exploram o monopólio construido na escassez a fim de maximizar o seu lucro. No Congo - país desértico e subdesenvolvido -, por exemplo, a industria mineradora de cobalto financia os setores essenciais como construção de moradias, produção de alimentos e apoio médico, além de empregar a maioria da população ativa. Com isso, a empresa mineradora estabelece com preços abusivos para as necessidades básicas, efetivamente forçando o cidadão a apoiar a extração de cobalto por ser sua única atividade capaz de sustentá-lo. Se a alternativa à exploração é sobreviver desempregado no deserto, a escassez se torna instrumento de vampirismo econômico.

Paralelo a isso, a abundância também corrobora a exploração. O filósofo sul-coreano Byung-Chul Han aponta que, nas sociedades do cansaço, mesmo com o aumento do valor da mão-de-obra dele decorrente, o capital é ferramenta de exploração. A título de ilustração, na Coreia do Sul - nação rica e desenvolvida - a Sansung oferece os melhores salários. No entanto, a aparente prosperidade oferecida pela empresa mascara o preço pago pelos funcinários em sangue, por essa riqueza. Nesse contexto, como indivíduos capitalistas buscam o lucro , eles se subvertem a exaustivas horas de estudo para adentrar à Sansung e, uma vez ingressas, são exploradas com insalubres ambientes e cargas horárias a fim de se manter “prósperos" e empregados. Na comcepção de Byung, essa situação configura em uma sociedade docente na qual as pessoas sacrificam a própria saúde mental e física para servirem a uma empresa que se beneficia mais com o trabalho delas do que elas mesmas. Configura-se, pois, uma sociedade do cansaço. Portanto, as vampiras empresariais exploram as sociedades ricas e canssadas com promessas de fartura paga pelo sangue.

Em suma, se o lucro é o objetivo, ele é atingido às custas do humano seja em contextos de carestia, seja em situação de abundância. Logo, faz-se imprescindível que se priorize o trabalhador ao invés do capital para combater a exploração. O sangue precisa valer mais do que o dolar.

\newpage
\begin{figure}
\centering
\includegraphics[width=0.48\linewidth]{CamScanner 08-02-2023 23.58_1.jpg}
\caption{\label{}A redação possui 501 palavras, e na folha de redação foram utilizadas 32 linhas}
\end{figure}
\section*{Comentários}
\begin{enumerate}
\item Tema: O texto instiga, mas ainda não concretiza uma aboradagem consistente, reflexiva e autoral sobre o tema; Ideias de textos da coletânea identificadas e utilizadas a partir de uma leitura crítico- interpretativa. 1,5|2
\item Tipo de texto: Tese com potencial argumentativo. Raciocínio completo que revela planejamento do texto por apresentar certa complexidade, mas com pontual quebra, salto ou lacuna que não compromete a progressão textual e a defesa da tese. Conclusão coerente com o desenvolvimento e projeto de texto, apresentando um desfecho efetivo para a conclusão. 1,5|2
\item Coerência: Compreesão satisfatória dos conceitos-chave. Não há incoerência externa, mas pode haver pontuais incoerêmcias internas que não comprometem da abordagem temática; Há ocorrência pontual de informações não suficientemente desenvolvidas; Pontuais imprecisões informativa
\item Coesão: Estratégica e acompanha a complexidade de sofisticação com êxito. 2|2
\item Modalidade: Poucos desvios, não prejudicam a leitura e compreensão das ideias. 1,5|2
\end{enumerate}

\newpage
\paragraph{Tema:} O \textit{Home Office} pós pandemia é vantajoso? 

\begin{flushright} Gustavo Cordeiro \end{flushright}
O Lar metamorfoseado e a totalização do trabalho

Já em meados do século XIX, o filósofo alemão Karl Marx identificava a formação do sistema capitalista, baseada na estruturação do trabalho como relação explorativa entre a burguesia e a classe trabalhadora. Em sua época, essa relação se dava exclusivamente nas fábricas e manufaturas, ambiente controlado pelo patronato e no qual se poderia vigiar disciplinarmente os trabalhadores. Entretanto, nos dias atuais, o avanço das tecnologias virtuais permitiram o advento do “Home Office", popularizado durante a pandemia e no qual a possibilidade de centralizar ainda mais a vida humana ao redor do trabalho, trazendo-o para o lar, fez com que se mantivesse mesmo após o fim do Covid-19. Sendo assim, essa reivenção do trabalho no capitalismo, vantajosa aos patrões, tem como objetivo conectar de maneira absoluta o trabalhador explorado às suas funções de trabalho, contaminando seu lar e isolando-o de seus pares.

Nesse sentido, pontua-se que o Home Office prejudica a independência moral do indivíduo em relação ao emprego de maneira ainda mais profunda que nos ambientes tradicionais de trabalho. Sobre isso, Marx descrevia as fábricas de seu tempo como lugares moldados para disciplinar e alienar a classe trabalhadora, por meio da vigília constante, horários rígidos e punições. Porém, ao transferir-se o ambiente de trabalho para a casa e flexibilizando todas essas condições, obtem-se um cenário ainda mais produtivo, no qual o indivíduo é induzido a acreditar na própria liberdade, ao mesmo tempo em que trasnforma o “lugar de viver" em local de trabalho. Tal ocorre, por exemplo,  quando se permite a flexibilização dos horários, mantendo altas metas; até mesmo os horários de lazer e sono passam a ser “colonizados" pelo trabalho. Portanto, o sistema usufrui da tecnologia para tornar ainda mais eficiente e , em especial, menos evidente a sua estrutura de exploração.

Com o mesmo obejetivo, o Home Office dificulta a mobilização dos trabalhadores e cristaliza as condições moldadas pelo patronato. Historicamente, nesse contexto, o convívio do operariado nas fábricas locais de trabalho foi fundamental na organizações de militâncias sindicais que lutaram pela instituição de direitos e concenssões trabalhistas. Atualmente, o isolamento causado pelo Home Office tende somente a enfraquecer essa mobilização, uma vez que o trabalhador não percebe a sua própria condição como maioria frente ao contato com seu semelhante. Têm-se, portanto, um cenário de cada vez maior de vulnerabilidade dos indivíduos nas condições de trabalho.


Em suma, vê-se que o Home Office pós-pandemia surge como um novo instrumento de exploração sistêmica. Sob o pretexto de liberdade e conforto, ele institui a totalização do trabalho mesclando-o com o lugar em que se vive, e alienando a classe trabalhadora para enfraquecê-la.


\newpage
\begin{figure}
\centering
\includegraphics[width=0.48\linewidth]{IMG-20230817-WA0017.jpg}
\caption{\label{}A redação possui 441 palavras, e na folha de redação foram utilizadas 35 linhas}
\end{figure}
\section*{Comentários}
\begin{enumerate}
\item Tema: O texto desenvolve as abordagens temáticas sugeridas e as amplia para algo que permite a caracterização de uma abordagem autoral e inovadora. Ideias de textos da coletânea identificadas e utilizadas de maneira produtiva e autoral; Uso pertinente e produtivo de repertório legitimado e articulado plenamente às explicações propostas na comprovação das ideias centrais. 2|2
\item Tipo de texto: Tese com potencial argumentativo. Raciocínio completo que revela planejamento do texto por apresentar certa complexidade, mas com pontual quebra, salto ou lacuna que não compromete a progressão textual e a defesa da tese. Conclusão coerente com o desenvolvimento e projeto de texto, apresentando um desfecho efetivo para a conclusão. 1,5|2
\item Coerência: Compreesão satisfatória dos conceitos-chave. Não há incoerência externa nem interna. 1,5|2
\item Coesão:Uso predominatemente correto e variado dos recursos de coesão. 1,5|2
\item Modalidade: Poucos desvios, não prejudicam a leitura e compreensão das ideias. 1,5|2
\end{enumerate}

\newpage
\paragraph{Tema:} De que maneira a exploração e a exclusão estruturam a \textit{desigualdade} social ?

\begin{flushright} Maurício Foletto \end{flushright}
       Elegia 2023

       
“Trabalhas sem alegria para um mundo caduco’’. Essa célebre frase começa “Elegia 1938” de Carlos Drummond Andrade, poema no qual o autor despeja toda sua angústia ante um mundo marcado pela desigualdade e critica a passividade do cidadão diante da sua dor irrefreável. Apesar de se tratar de uma abra quase centenária, tais versos caracterizam de forma precisa a situação do mundo atual, local no qual, mais do que nunca, a exploração e a exclusão são ferramentas de segregação numa sociedade apacificada. Nesse sentido, o abuso pregado pelo sistema econômico vigente impede a mudança desse cenário enquanto propulsiona uma alienação supressora entre hierarquias.


De início, se o capitalismo tem, por base, a diferenciação de classes, a exploração se torna uma máxima a ser seguida. De acordo com Karl Marx, a “Grande Máquina” escrita por Drummond se alicerça na formação de uma classe massificada de trabalhadores descartáveis e outra, reduzida, de detentores dos meios de produção que, no fim do processo mercantil, fica com quase todo o lucro para si. Dessa maneira, o proletariado explorado não consegue o acúmulo de dinheiro prometido pelo capitalismo e, por isso, possui menos tempo para dedicar ao estudo e ao desenvolvimento do capital cultural que tornaria sua mão de obra menos informal e substituível, uma vez que deve se auto explorar trabalhando cada vez mais para manter seu lugar no mercado e não ser descartado. Com isso, quando impede a ascensão social, a exploração se retroalimenta e mantem a desigualdade, como observado com no “sticky floor” (chão grudento), da Oxford, estudo que analisa as dificuldades que famílias têm em mudar de jerarquia social, sendo as de mais baixa renda as mais propensas a ficarem grudadas em sua situação. Dessa forma, e tal qual Drummond dita, o indivíduo “pratica laboriosamente os gestos universais” exploratórios que o são impostos pela “Grande Máquina”.


Consequentemente, conforme a concentração de renda se estabelece, o proletariado é coagido à exclusão. Segundo Marx, o capitalismo só consegue existir pela alienação do proletariado -detentor apenas da sua força de trabalho- diante da sua situação enquanto engrenagem da máquina econômica para que, assim, não haja revolta populacional. Isto posto, nasce uma economia que tenta excluir o trabalhador não apenas da ascensão entre hierarquia sociais, mas da própria ideia de mudança para que, com isso, o cidadão, nas palavras de Drummond, “confesse sua derrota e adie para outro século a felicidade coletiva” que viria com o desmantelamento do capitalismo. A título de exemplo toma-se a implementação do novo ensino médio nas escolhas públicas que, ao passo que elimina matérias importantes para a formação de criticidade – como filosofia e história-, apresentam aulas de “brigadeiro artesanal” nas disciplinas de investimento, enquanto, nas escolas particulares, tais aulas explicam o modo de lucrar na bolsa de valores, demonstrando a tentativa de alienação dos alunos e a exclusão de chances de os pobres. Destarte, o sistema econômico em si faz o indivíduo, para Carlos, aceitar a chuva, a guerra, o desemprego e a injusta distribuição.


A exploração e a exclusão são, portanto, peças importantes na “grande máquina” geradora de desigualdades. Isso porque, no momento que uma impede a mudança da situação a outra tira a esperança e alegria do trabalhador pois, para Drummond, “não se pode, sozinho, dinamitar a ilha de Manhattan”. 


\newpage
\begin{figure}
\centering
\includegraphics[width=0.48\linewidth]{Mauricio_1.jpg}
\caption{\label{}A redação possui 545 palavras, e na folha de redação foram utilizadas 33 linhas}
\end{figure}
\section*{Comentários}
\begin{enumerate}
\item Tema: Uso pertinente e produtivo de repertório legitimado e articulado plenamente às explicações propostas na comprovação das ideias centrais; As palavras-chaves do tema aparecem na introdução e nos dois desenvolvimentos; Sem uso de senso comum acompanhado de aprofundamento da coletânea. 2|2
\item Tipo de texto: Tese apresenta todas as palavras-chaves e com ponto de vista com bom potencial argumentativo; Raciocínio configurado de forma sofisticada, analogia e referências teóricas determinam toda a condução argumentativa; Argumentos autorais e variados. 2|2
\item Coerência: Compreesão satisfatória dos conceitos-chave. Não há incoerência externa nem externa. 1,5|2
\item Coesão:Uso predominatemente correto e variado dos recursos de coesão. 1,5|2
\item Modalidade: Poucos desvios, não prejudicam a leitura e compreensão das ideias; A leitura é fácil não porque o conteúdo seja superficial, mas porque a escolha e a combinação de palavras, assim como as estruturas sintáticas, atuam para transmitir com clareza e precisão as ideias; Marcas estilísticas, como figuras de linguagens, são utilizadas com segurança e contribuem ativamente para a argumentação. 1,5|2
\end{enumerate}
\newpage
\section*{Redações nos vestibulares} \paragraph{ ITA 2023:}A responsabilidade da engenharia frente aos problemas do mundo contemporâneo.
\begin{flushright} Marco Antônio \end{flushright}
Dual: responsável pela vida e pela morte

Segundo o filósofo Michel Foucault, em sua obra “A ordem do discurso", a construção do saber se dá por meio da arqueologia do conhecimento, ou seja, na formação axiomática e em sua progressão pragmática. Nesse viés, Foucault analisa como a força do conhecimento sustenta o discurso do poder vigente e como este é utilizado na gestão de vidas humanas, cunhando o termo “Biopolítica". Assim, ao observar a responsabilidade da engenharia nos hodiernos dias, nota-se que ela não apenas viabiliza a apoteose do desenvolvimento tecnológico, mas também na gestão de vidas e na manutenção do poder dominante; caracterizando, pois, a engenharia como um fator crucial na biopolítica foucaultinana aos atuais problemas socio-políticos.

Em primeiro plano, em uma sociedade hipermoderna o conhecimento é coarctado a uma mercadoria e, por conseguinte, é negociado entre países por meio de pessoas. Consoante o sociólogo Gilles Lipovetsky, em sua obra “Era do Vazio", o termo hipermodernidade caracteriza o tecido social no qual qualquer âmbito das relações humanas se transmutam em mercadoria. Nesse prisma, o conhecimento é comercializado por meio do que hoje é conhecido como “Fuga de Cérebros", que ocorre quando um polo econômico fomenta acoagulação de estudiosos, o que resulta em um monopolo dententor de grandes informações. A título de exemplo, têm-se o projeto Manhattan ocorrido em 1945 no qual concentrou dezenas de cientistas ao redor do globo a fim de desenvolverem uma arma de guerra - a bomba atômica - que culminou o final da $2^a$ Gerra Mundial. Dessa forma, evidencia-se como um polo econômico como os EUA utilizou-se da engenharia para endossar seu poder político e bélico,  por meio da movimentação e incentivo financeiro de indivíduos com o domínio do conhecimento da engenharia. Logo, o caráter hipermoderno da sociedade é evidenciado na transformação do saber científico em uma mercadoria para garantir a manutenção do poder.

Consequentemente, garantido o poder político por meio da posse dos saberes da engenharia, tais conhecimentos responsabilizam-se pela gestão da vida dos indivíduos. Isso ocorre devido ao fato da engenharia garantir a melhora dos padrões de vida; quando, a mesma, utilizada para fins bélicos responsabiliza-se não apenas pela gestão da vida, mas principalmente pela gestão da morte; ressemantizando-a para uma asseguradora não mais da biopolítica, mas sim da necropolítica. Esse contraste da responsabilidade dual da engenharia pode ser observados pela imagem dos engenheiros e cientistas Harber-Bosch que fundamentaram a rota sintética da produção de amônia a qual corroborou tanto a produção de fertilizantes - possibilitando o exponencial crescimento humano no século XX - quanto na utilização de explosivos - utilizados até mesmo por R. Oppenheimer como estopim na ativação da bomba atômica, levando a milhares de morte no Japão. A vida é, pois, influenciada pelo controle e domínio da engenharia.

Portanto, a responsabilidade da engenharia frente aos problemas contemporâneos é dual, já que viabiliza tanto o desenvolvimento científico quanto a gestão de vidas e mortes. Por isso, em uma sociedade hipermoderna, os detentores do conhecimento tecnocientífico da engenharia possuem, por conseguinte, podeiro político e bélico.


\begin{figure}
\centering
\includegraphics[width=0.48\linewidth]{IMG-20240124-WA0018.jpg}
\includegraphics[width=0.48\linewidth]{IMG-20240124-WA0017.jpg}
\caption{\label{}A redação possui 507 palavras, e na folha de redação foram utilizadas 33 linhas, tempo utilizado: 1h30min direto na folha de redação - sem rescunho. }
\section*{Pontuaçã: 8,75}

\subsection*{Notas por critério:}
\begin{enumerate}
 \item 2,00 \item 1,50  \item 2,00 \item 1,75 \item 1,50

\end{enumerate}
\end{figure}

\newpage 
\paragraph{ Fuvest 2023:}Educação básica e formação profissional: entre a multitarefa e a reflexão.

\begin{flushright} Marco Antônio \end{flushright}

O Discurso da hiperprodutividade

Segundo o filósofo Michel Foucault, em sua obra “A ordem do discurso", o discurso associado ao saber tende a ser utilizado de modo a garantir a manutenção do poder vigente. Nesse perspectiva, ao analisar a sociedade contemporânea, nota-se que as bases para a construção dos discursos e dos conhecimentos - a educação básica - está intrinsecamente atrelada à lógica de mercado. Desse modo, tal lógica corrobora a alienação de uma sociedade cada vez mais capitalista em que o ócio é desvalorizado em detrimento da apoteose da produtividade; o que aturde o pensamento crítico e a reflexão social à um segundo plano. Isso transforma, por fim, os indivíduos em “animais-laborans" em que a multitarefa se coarcta na busca pelo arrivismo social.

Em primeiro plano, em uma sociedade hipermoderna a educação básica visa exclusivamente o preparo de um cidadão voltado para a formação profissional, posto que o discurso neoliberal sustenta a hodierna estrutura de poder. Consoante o sociólogo Gilles Lipovetsky, em sua obra “Era do Vazio", o termo hipermodernidade caracteriza todo o tecido social no qual quais quer âmbito das relações humanas possam se transfigurar em mercadoria. Dessa forma, posto o caráter pós-moderno social, as instituições de ensino viabilizam a manutenção da estrutura capitalista ao difundirem um ideal de produtividade. Esse ideal pode ser observado tanto pela visão foucaultinana na qual a própria infraestrutura de ensino exercer coerção sobre o indivíduo  - por meio do sistema de auto vigilância:“panoptico" -, tanto pela visão lipovetskysiana em que a própria formação escolar é transmutada em mercadoria, visto seu prognóstico exclusivo para o mercado de trabalho. Logo, percebe-se um ofuscamento do ócio mediante a ascensão de discursos neoliberais que catalisam a transfiguração da formação profissional e educacional em objeto de mercado.

Por conseguinte, a desvalorização do ócio e a superestimação do trabalho levam à alienação do indivíduo à ideologia da hiperprodutividade. Isso ocorre devido à solidificação de discursos arrivistas amplamente divulgados em mídias digitais como You Tube e Instagram, a título de exemplo a frase:“ Trabalhe enquanto eles dormem, e viva o que eles sonharam ". Tais discursos influenciam o crescimento de uma sociedade cada vez mais cansada na qual se propicia o aumento de doenças como burnout - esgotamento físico e mental - contribuindo, pois, ainda mais para a alienação social. Assim, a sociedade, uma vez alienada, amplia o atributo da multitarefa dos indivíduos, já que estes se transmutaram em “animais-laborans".

Portanto, nota-se que em uma sociedade hipermoderna a educação básica fomenta unicamente a formação profissional. Isso se deve ao fato da desvalorização do ócio e da reflexão em prol da proliferação do discurso neoliberal, que assegura a manutenção da estrutura capitalista vigente de poder; poder esse que induz à hiperprodutividade e aturde os cidadãos à multitarefas arrivistas.

\newpage 
\begin{figure}
\centering
\includegraphics[width=0.48\linewidth]{IMG-20240122-WA0007(1).jpg}
\caption{\label{}A redação possui 457 palavras, e na folha de redação foram utilizadas 29 linhas, tempo utilizado: 2h15 considerando rascunho e reescritura
(perceba que eu comecei escrevendo em cima da linha numero 1) }

\section*{Nota: 48/50}
\end{figure}.





\newpage
\paragraph{ Fuvest 2023:}Educação básica e formação profissional: entre a multitarefa e a reflexão.

\begin{flushright} Augusto Esper  \end{flushright}

Poder, Reflexão e Sociedade 

Segundo Michel Foucault na obra “A Ordem do Discurso", o discurso tende a ser utilizado de modo a regular a sociedade. Ao estabelecer a relação entre poder e educação, Foucault propõe uma nova definição, que admite que o sistema educacional moderno é recorrentemente aparelhado de modo a coibir a reflexão e, consequentemente, de modo a manter a estrutura social intacta. Ele argumenta que esse fenômeno é marcado por uma educação voltada para o mercado de trabalho, que valoriza a multitarefa. Logo, além de entender os malefícios de um cenário educacional voltado para o trabalho e para a multitarefa, é fundamental compreender a necessidade de se estimular a reflexão durante a educação básica.

Em primeira análise, em uma sociedade, o poder é intrinsecamente ligado ao controle do ideário social. Conforma Foucault, a disciplinação ocorre através da linguagens que se proliferam indefinidamente, caracterizando a Sociedade do Discurso.  Desse modo, é evidente como historicamente a educação básica foi utilizada como mecanismo disciplinador, como na Alemanha Nazista, onde a exibição de ideias preconceituosas nas escolas consolidou a sinpatia de diversas crianças ao regima. No século XXI, o tecido social se modificou, logo, o sistema educacional é frequentemente construído para atender as necessidades do mercado, como na Coreia do Sul, onde uma educação básica pautada na produtividade e na multitarefa é amplamente atrelada aos altos índices de depressão no país. Então, evidencia-se os malefícios de uma educação fortemente ligada ao mercado de trabalho.

Por conseguinte, na Sociedade do Discurso, narrativas construídas em ambientes escolares são consideradas mais importantes do que a evidência e verdade factual. Esse contexto esclarece o conceito foucaultiano de “pós-verdade", que retoma a relação entre poder e educação cujo objetivo é a dominação do homem a partir da construção da “verdade". Desse modo, é nítido que ao se incentivar a capacidade de reflexão das crianças na educação básica, potencializa-se a formação de cidadãos autônomos que ofusquem o poder dominador descrito por Foucault. Tal teoria é observada nas escolas suecas, que são voltadas para a formação humanitária e reflexiva da criança acerca de seu papel na sociedade. Torna-se clara a importância  de estimular a reflexão durante a educação básica.

Inferes-se, portanto, que entre a multitarefa e a capacidade reflexiva, é fundamental priorizar a habilidade de refletir no sistema educacional. Assim, ocorrerá a formação de adultos para a convivência em sociedade.

\begin{figure}
\centering
\includegraphics[width=0.48\linewidth]{IMG-20240124-WA0021.jpg}
\caption{\label{}A redação possui 396 palavras, e na folha de redação foram utilizadas 30 linhas,}

\section*{Nota: 42/50}
\end{figure}.




\maketitle
\newpage
\section*{Michel Foucault}
\begin{flushright} 1926-1984 | França \end{flushright}

\begin{itemize}
    \item Metodologia: Genealogia e Arqueologia do Saber
    \item Objeto de Estudo: Poder e Controle Social
\end{itemize}
   
    Seu pensamento consiste na tese da ausência de verdades evidentes, todo o saber foi produzido em algum lugar, com um propósito. Por isso mesmo, pode ser criticado, transformardo, e, até mesmo destruido. Além disso, suas ideias acompanham engajamento em uma tarefa política buscando a analise das estruturas de dominação, coerção e manipulação do comportamento humano.

    Principais obras abordam, de forma simplificada: O nascimento dos Hospitais; As mudanças no espaço arquitetural que servem para punir, vigiar, separar; O uso da informação e veículos de mídia para o controle da população; Como e por que a sexualidade passa ser alvo de preocupação médica e sanitária; Como governar seignifica gerenciar vidas( Biopolítica - Necropolítica ), e tornar os indivíduos mais produtivos, sadios e governáveis.


 \subsection*{Principais Obras}


\subsubsection*{Vigias e Punir: 1975}


O livro aborda a temática dos mecanismos de poder sobre os indivíduos.
\vspace{1mm}

Os corpos são controlados por poderes e interesses externos às pessoas que domesticam outrem.
\vspace{1mm}

Todas as Instituições Sociais produzem o mesmo padrão de controle social dos corpos (escolas, hospitais, prisões, alojamentos)
\vspace{1mm}

Sociedade Diciplinar: As Instituições e a arquitetura são pensadas para controlar o ente social. Origina o conceito de Panoptico que é representado como uma prisão circular em que um guarda central tem visão completa dos presos, mas os presos não têm visão uns dos outros ou do guarda. Esse modelo sugere que a possibilidade de ser observado constantemente é suficiente para que os indivíduos internalizem a disciplina e auto-regulem seu comportamento, sem a necessidade de intervenção direta do guarda. Para Foucault, o panóptico é uma metáfora para a forma como o poder é exercido na sociedade moderna, através de mecanismos de vigilância e controle social - de modo camufaldo, sutil, micro.
\vspace{1mm}


“Muitos processos disciplinares existiam há muito tempo: nos 
conventos, nos exércitos, nas oficinas também. Mas as disciplinas 
se tornaram no decorrer dos séculos XVII e XVIII fórmulas gerais de 
dominação. (…) O momento histórico das disciplinas e o momento 
em que nasce uma arte do corpo humano, que visa não unicamente 
ao aumento de suas habilidades, nem tampouco aprofundar sua 
sujeição, mas a formação de uma relação que no mesmo mecanismo 
o torna tanto mais obediente quanto é mais útil, e inversamente. 
Forma-se então uma política das coerções que são um trabalho 
sobre o corpo, uma manipulação calculada de seus elementos, de 
seus gestos, de seus comportamentos. O corpo humano entra numa 
maquinaria de poder que o esquadrinha, o desarticula e o recompõe. 
Uma “anatomia política”, que é também igualmente uma “mecânica 
do poder”, está nascendo; ela define como se pode ter domínio sobre 
o corpo dos outros, não simplesmente para que façam o que se quer, 
mas para que operem como se quer, com as técnicas, segundo a 
rapidez e a eficácia que se determina. A disciplina fabrica assim 
corpos submissos e exercitados, corpos “dóceis”. (Vigiar e Punir, p. 
118)
\begin{multicols}{2}
\subsection*{História da Loucura - 1961}
Aborda  controle e classificação e poder sobre os corpos, os 
saberes, o uso da medicina para o poder, o Normal, o Anormal e a 
Exclusão, Instituições Sociais. 
 
 Normal – aquilo que é padronizado. É base para julgar o anormal. 
 
 O julgamento é naturalizado – acusar/evidenciar o anormal. 

 Prisão e o hospício – afastamento de quem não é normal – de 
quem não é produtivo. 


O saber dá 
elementos de controle ao poder; quem possui esses saberes ganha 
poder; o poder investe na construção desses saberes. O saber tem 
um poder de “efeito de verdade” - o poder faz o saber parecer uma 
verdade. Quem não tem poder, não tem seu saber valorizado. 

 Medicalização da vida – não se espera que uma pessoa passe 
mal no trabalho e atrapalhe a produção devido a problemas de 
saúde. Logo, a medicina é utilizada para que o trabalhador seja 
produtivo e não para cuidar da saúde do trabalhador.


\subsection*{Microfísica do Poder-1970}
O poder não é algo que uma instituição, pessoa ou grupo possua de maneira estática e controladora, mas sim uma relação dinâmica e disseminada em todas as esferas da sociedade. Foucault argumenta que o poder está presente em todas as relações sociais e é exercido de forma descentralizada, agindo em diversos níveis e microestruturas do cotidiano.

Ele propõe uma análise do poder que vai além das estruturas de dominação óbvias, como o Estado ou instituições políticas, e explora como o poder opera no nível individual e nas interações sociais diárias. Foucault descreve o poder como uma rede complexa de relações e práticas sociais, envolvendo formas de controle, disciplina, vigilância e resistência.

Para Foucault, o poder não é apenas repressivo, mas também produtivo, ou seja, ele cria e molda normas, comportamentos e identidades sociais. Ele enfatiza a importância de estudar as tecnologias do poder, como as instituições, discursos, conhecimentos e práticas que permitem o exercício do poder sobre os indivíduos.

Portanto, a obra busca analisar como o poder se manifesta no nível mais íntimo das relações sociais, influenciando a subjetividade e a forma como os indivíduos se percebem e se comportam em uma sociedade.

\subsection*{A Ordem do Discurso - 1970}
 É uma obra seminal que desvela a intrincada relação entre linguagem, poder e conhecimento. Por meio de uma abordagem arqueológica do conhecimento, Foucault mergulha nas profundezas da construção histórica do discurso, revelando uma série de ideias e conceitos centrais que lançam luz sobre a natureza do saber e sua influência na sociedade.

Um dos conceitos-chave apresentados por Foucault é o da "episteme", que se refere a um sistema de regras e estruturas que orientam o pensamento e a produção de conhecimento em uma determinada época. Segundo Foucault, cada período histórico possui sua própria episteme, o que implica que o conhecimento não é uma entidade estática, mas sim uma construção profundamente enraizada no contexto cultural e histórico.

A obra também enfatiza a inseparabilidade entre discurso e poder. Foucault argumenta que o discurso não é apenas uma forma de expressar ideias, mas também uma ferramenta de exercício de poder. As instituições e autoridades detêm o controle sobre o discurso, moldando não apenas o que é dito, mas também quem tem permissão para falar e em que termos. Nesse sentido, o sujeito do discurso não é um agente autônomo, mas uma construção social moldada pelas normas do discurso em vigor.

A sociedade se disciplina através da linguagem das ideias que se
proliferam indefinidamente, caracterizando a Sociedade do Discurso.


\end{multicols}
\newpage
\section*{Guy Debord}
\begin{flushright} 1931-1994 | França \end{flushright}
\begin{itemize}
    \item Metodologia: Análise soiocomportamental
    \item Objeto de Estudo: Ferramentas de manipulação sociopolítica 
\end{itemize}
O ponto central de sua teoria é que a alienação é mais do que uma descrição de emoções ou um aspecto psicológico individual. É a conseqüência do modo capitalista de organização social que assume novas formas e conteúdos em seu processo dialética de separação e retificação da vida humana, em uma sociedade cuja experiência dissocia-se da propria realização.

\subsection*{Principal Obra}
\subsection*{A Sociedade do Espetáculo | 1967}
A rigor, pode-se afirmar, uma premissa constitutiva
desse texto: o espetáculo como um momento e um movimento imanentes
à vida societária, de maneira similar às encenações, aos ritos, rituais, imaginários, representações, papéis, máscaras sociais, entre outros. Portanto,
o espetáculo deve ser compreendido como inerente a todas as sociedades
humanas e, por conseguinte, presente em praticamente todas as instâncias organizativas e práticas sociais, dentre elas o poder político e a política.

Ainda que, em um livro, escrito ao estilo manifesto, seja difícil exigir e
buscar um conceito rigoroso e nitidamente formulado sobre ele, dois eixos
interpretativos ganham destaque e podem servir de âncora para compreender
a concepção de espetáculo, conforme a construção teórica de Debord.
Um desses eixos aponta o espetáculo como expressão de uma situação histórica em que a “mercadoria ocupou totalmente a vida social”
(DEBORD, 1997, p. 30). Espetáculo, mercadoria e capitalismo estão
umbilicalmente associados. Desse modo, a sociedade do espetáculo pode
ser interpretada como conformação avançada do capitalismo, como a etapa
contemporânea da sociedade capitalista.

O outro eixo interpretativo, que interessa sobremodo à escritura deste
texto, é a anunciada separação entre real e representação. Tal cisão, consumada na contemporaneidade, inaugura a possibilidade da sociedade do
espetáculo. Nela, as imagens passam a ter lugar privilegiado no âmbito
das representações. Nas palavras de Debord, “O espetáculo, como tendência a fazer ver (por diferentes mediações especializadas) o mundo que
já não se pode tocar diretamente, serve-se da visão como sentido privilegiado da pessoa humana” (DEBORD, 1997, p. 18). Entretanto, “o espe-
táculo não é um conjunto de imagens, mas uma relação social entre pessoas, mediada por imagens” (DEBORD, 1997, p. 14). A emergência de
uma sociedade do espetáculo depende, assim, desta “separação consumada”, mas requer uma outra condição: a autonomização da representação
frente ao real. “Sempre que haja representação independente, o espetáculo se reconstitui” (DEBORD, 1997, p. 18).

Com finalidade disertativa-argumentativa, um grande exemplo no qual o espetáculo se consuma está nas redes sociais, nas quais tornam-se vitrines, e o usuário transfigura-se em consumidor - Ideia de uma sociedade hipermoderna espetacularizada.

\newpage
\section*{Gilles Lipovetsky}
\begin{flushright} 1944 - ??? (78 anos) | França \end{flushright}
\begin{itemize}
    \item Metodologia: Projetual
    \item Objeto de Estudo: Pós-modernidade e Hipermodernidade
\end{itemize}
\hspace{6mm}Suas principais concepções orbitam em torno da sociedade do hiperconsumo e da incessante busca pela felicidade individual. Argumenta-se que vivemos em uma era caracterizada pelo individualismo exacerbado, no qual o consumo desenfreado, a busca incessante pela juventude eterna e a valorização do prazer instantâneo assumem papel central. Ele explora detalhadamente como essa cultura efêmera e hedonista influencia a política, a economia e as interações interpessoais. As obras de Lipovetsky desempenham um papel fundamental na compreensão das transformações sociais e culturais ocorridas nas últimas décadas, desafiando premissas tradicionais acerca da sociedade e da condição humana.

\subsection*{Principais Obras}


\subsection*{A Era do Vazio - 1988}
Um ensaio sobre o Individualismo contemporâneo.


"[...] o vazio representa um novo conteúdo. A modernidade estruturou-se como imaginário do 
dever e do homogêneo. Cada indivíduo precisava corresponder ao imperativo moral 
dominante, mesmo naquilo que só dizia respeito ao espaço privado. A ideia de imperativo 
serviu de cobertura para a imposição de visões de mundo e para a exclusão de todos os que 
ousaram postular modos alternativos de vida [...] a comunicação como forma de contato, 
expressão de desejos, emancipação do jugo utilitário."


O indivíduo hipermoderno é portador de uma personalidade narcisista, seus desejos 
individualistas passam a ter mais valor do que os desejos e interesses de grupos,enfraquecendo os movimentos sociais e a vida coletiva. Esse novo indivíduo tem uma busca 
incansável de si mesmo, desprendendo-se de vez do domínio do outro. 

"[...] amar a mim 
mesmo o bastante para não precisar de outra pessoa para me fazer feliz". (Lipovetsky, 2005, 
p. 36).


Neste novo cenário existe a ilusão de sociabilidade e cooperatividade, porém é apenas 
aparente, por trás da tela do hedonismo e da solicitude, cada um explora cinicamente os 
sentimentos dos outros e satisfaz seus próprios interesses sem se preocupar com as gerações 
futuras, afundando cada vez mais o homem no vazio. Atividades cotidianas como comunicar-
se, ouvir e ser ouvido tornaram-se obrigações, o isolamento do homem como ser social é a 
valorização do ser individual.


"[...] quanto mais se desenvolve as possibilidades de encontro, mais os indivíduos se 
sentem sós; quanto mais as relações se tornam livres, emancipadas das antigas 
restrições, mais rara se torna a possibilidade de conhecer uma relação intensa. Por 
todo lado há solidão, vazio, dificuldade de sentir, de ser transportado para fora de si 
mesmo." (Lipovetsky, 2005, p. 57).


Essa necessidade e urgência de sentimentos, prazer e satisfação efêmera, características do 
indivíduo hipermoderno, nos lembra um pouco as lições do Dalai Lama "vivem como se 
nunca fossem morrer e morrem como se nunca tivessem vivido". E parafraseando Bauman 
(2001, p.47). "[...] nada permanece parecido, imutável, durante muito tempo, nada dura o 
suficiente para se tornar familiar, acolhedor e tranquilo." 
\newpage
\begin{multicols}{2}

   \subsection*{Os Tempos Hipermodernos}
A hipermodernidade é a era pós-moralista, o fim de uma época de valorização do sacrifício e 
de condenação do prazer, a queda de uma moral rigorista que fazia do homem o chefe da 
família, a autoridade paterna, a voz incontestável, o esteio da sociedade dentro do 
microcosmo do lar e a mulher em uma situação secundária sem direito a voz, é o surgimento 
de uma era polissêmica, que trouxe o desprendimento das tradições, do sentido histórico e das 
preocupações exacerbadas com o futuro.


"[...] estamos menos carregados e mais livres, mais lúcidos e menos dependentes, mais 
exigentes e menos submissos, mais flexíveis e menos engessados por engrenagens de 
poder em nome de verdades que se apresentavam como transcendentais ou universais, 
embora não passassem de formas locais de controle." (SILVA, 2005, p. X).


Esses novos tempos trazem a libertação e abertura, onde o indivíduo recusa os esquemas 
impositivos, tem o direito de ser simplesmente ele mesmo, de satisfazer-se, de ter a sua 
subjetividade respeitada, ser único e viver o hoje, o agora com urgência. Nesse contexto nasce 
o individualismo e com ele uma nova manifestação narcisista.


"Não queremos a ilusão do futuro nem a coerção do passado. Postulamos a 
intensidade do aqui e do agora como necessidades vitais. Não aceitamos viver de 
promessas nem de patrimônio acumulado. Exigimos fazer por nós mesmos o que 
somos e o que seremos, sem garantias de redenção nem obrigações inquestionáveis."
(Silva, 2005, p.XIII)

Dessarte, simplificadamente a sociedade hipermoderna é caracterizada pelas constantes transfigurações de quaisquer âmbito das relações humanas e sociais em mercadoria. 
\vspace{85mm}
\subsection*{O Império do efêmero - 1989}
   O livro retrata a apoteose  da  sedução, a  publicidade libertou-se da racionalidade argumentativa,  pela  qual  se  obrigava  a  declinar  a  composição  dos  produtos,  segundo uma  lógica  utilitária,  e  mergulhou  num imaginário  puro,  livre  da  verossimilhança, aberto  à  criatividade  sem  entraves,  longedo  culto  da  objetividade  das  coisas. 
   
   Basicamente, ele reforça a comcepção da pós-verdade na qual a verdade factual (axiomática e sólida) e a evidência objetiva são muitas vezes consideradas menos relevantes do que as narrativas persuasivas e emocionalmente cativantes. Assim, os indivíduos são facilmente manipulados e seduzidos por artifícios retóricos, que ocasiona a sedução para o consumo - o vício e a efeméride.

\end{multicols}

\end{document}
